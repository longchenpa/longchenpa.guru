\documentclass{article}
\usepackage{indentfirst}
\usepackage{listings}
\usepackage{amsmath}
\usepackage{amssymb}
\usepackage{amsthm}
\usepackage[english,ukrainian]{babel}
\usepackage{url}
\usepackage[utf8]{inputenc}
\usepackage[T1]{fontenc}
\include{journal}

\begin{document}

\title{Дві істини}
\author{Дза Патрул $^1$}
\date{ $^1$ Лонгчен Нінгтік \\ \today }

\maketitle

\begin{abstract}Настанова щодо погляду Махаяни: Роз’яснення двох істин. \\
\textbf{Keywords}: Махаякна
\end{abstract}

\ifincludeTOC
  \tableofcontents
\fi

\newpage

\section{Вчення про те, що потрібно усвідомити}

\subsection{Природний стан усіх явищ, що можна пізнати}

\subsubsection{Відносний аспект}

\subsubsection{Абсолютний аспект}

\subsection{Природний стан власного розуму}

\subsubsection{Тимчасове розуміння в термінах двох істин}

\subsubsection{Отаточне розуміння, в якому істини є неподільними.}

\section{Вчення про те, як це втілити на практиці}

\subsection{Пряма практика для тих, що із найгострішими здібностямми}

\subsection{Поступова практика для тих, що із менш гострими здібностями}

\end{document}