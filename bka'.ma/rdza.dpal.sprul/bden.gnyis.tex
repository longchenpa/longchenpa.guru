\documentclass{article}
\usepackage[english,ukrainian]{babel}
\usepackage[utf8]{inputenc}
\ProvidesPackage{dharma}
\usepackage{caption}
\usepackage{hyperref}
\usepackage{url}
\usepackage[utf8]{inputenc}
\setlength{\parindent}{15pt}
\newif\ifincludeTOC
\includeTOCtrue
\lefthyphenmin=1
\hyphenpenalty=100
\tolerance=11000
\emergencystretch=1em
\hfuzz=2pt
\vfuzz=2pt


\begin{document}

\title{Дві істини}
\author{Дза Патрул $^1$}
\date{ $^1$ Лонгчен Нінгтік \\ \today }

\maketitle

\begin{abstract}Настанова щодо погляду Махаяни: Роз’яснення двох істин. \\
\textbf{Keywords}: Махаякна
\end{abstract}

\ifincludeTOC
  \tableofcontents
\fi

\newpage

Для тих, хто прагне досягти звільнення, є два вчення: (1) вчення про те, що потрібно усвідомити, і (2) вчення про те, як це втілити на практиці.

\section{Вчення про те, що потрібно усвідомити}

У цьому розділі розглядаються два аспекти: (1.1) природний стан усіх явищ, що можна піз нати, загалом і (1.2) природний стан власного розуму.

\subsection{Природний стан усіх явищ, що можна пізнати}

Цей розділ також поділяється на два аспекти: (1.1.1) відносний і (1.1.2) абсолютний.

\subsubsection{Відносний аспект}
Загалом, усі явища — від найнижчого пекла Найвищого Страждання до післямедитаційного досвіду бодгісатв на десятому бхумі включно — є відносними. Більше того, є два види відносного: неправильне відносне і правильне відносне. Усе, що ми сприймаємо до того, як ступаємо на шлях, належить до категорії неправильного відносного. Коли ми досягаємо етапу «прагнення до поведінки» [1], якщо ми можемо інтегрувати певне усвідомлення у наш досвід, це стає правильним відносним, але коли ми цього не робимо, це залишається неправильним відносним.

Після досягнення бхум усі явища, що постають перед  розумом, є правильним відносним — «відносним», тому що «просто явища» ще не припинилися, і «правильним», тому що їхня хибність сприймається безпосередньо. Ці явища продовжують виникати від першої до десятої бхуми, оскільки давня звичка сприймати речі як реальні ще не була подолана, подібно до того, як аромат мускусу залишається в посудині. Зрештою, на рівні будди, коли ці звичні тенденції повністю викорінено, дуалістичних сприйнять більше немає, і людина перебуває виключно в абсолютній сфері, поза будь-якими концептуальними ускладненнями. Прив’язаність до звичайного світу, як до зовнішнього середовища, так і до істот у ньому, як до реального, є неправильним відносним. Протиотрутою до цього, наприклад, є уявлення всіх як чистих божеств і середовища як чистого палацу мандали, водночас вважаючи їх лише ілюзією — це правильне відносне.

\subsubsection{Абсолютний аспект}

По суті, абсолют — це основний простір явищ (дгармадгату), вільний від усіх концептуальних ускладнень. У своїй сутності він не має поділів, але все ж можна говорити про «поділи» залежно від того, чи була ця реальність усвідомлена. Таким чином, є поділ на абсолют, який є самою основною природою, і абсолют, який є усвідомленням (або «проявленням») цієї основної природи. Також є поділ на абсолют, який прояснюється [2] через вивчення і роздуми, і абсолют, який переживається через медитаційну практику; або абсолют, який концептуально виводиться звичайними істотами, проти абсолюту, який безпосередньо переживається шляхетними істотами. Існує також поділ на концептуальний абсолют (намдрангпе дьондам) і абсолют, що перебуває поза концептуалізацією (намдранг майінпе дьондам). Є три способи, якими ми можемо переживати ці дві істини:

\begin{itemize}
\item На етапі звичайних істот явища вважаються реальними за своєю природою і сприймаються з прив’язаністю. Це називається неправильним відносним.
\item На етапі шляхетних істот явища усвідомлюються як оманливі і сприймаються без прив’язаності. Це називається правильним відносним.
\item На етапі будди немає ні звичайних явищ, ні їх відсутності, і будь-які турботи про прив’язаність чи неприв’язаність більше не застосовуються. Це називається абсолютним.
\end{itemize}

Іншими словами, на першому етапі є і явища, і прив’язаність, на середньому етапі є лише явища без прив’язаності, а на останньому етапі немає ні явищ, ні прив’язаності. Ці три етапи також відомі як «неправильне знання», «знання розуміння відносного» і «знання розуміння абсолютного». У випадку звичайних істот мудрість розуміння відносного залежить від аналізу, але для шляхетних істот вона досягається через безпосереднє сприйняття. Хоча звичайні поняття, такі як «розуміння» чи «нерозуміння», не застосовуються до самого абсолютного простору реальності, ми все ще можемо використовувати терміни, такі як «розуміння» чи «усвідомлення», щоб позначити визнання цього стану. Зрештою, ми повинні усвідомити неподільність двох істин, але твердження, що відносне стосується існування, тоді як на абсолютному рівні речей не існує, ніколи не вважатиметься поглядом Серединного Шляху. Коли ми усвідомлюємо єдину справжню природу правильного відносного, дві істини зливаються нерозривно, поза концептуальними крайнощами існування, неіснування, сталості чи порожнечі. Як сказано в «Матері Праджняпараміти»:

Справжня природа відносного є справжньою природою абсолютного.

Поділ на дві істини є лише умовним прийомом, заснованим на різних перспективах двох станів розуму, що використовується для полегшення розуміння. Усі різноманітні сутності, які постають перед спантеличеним станом розуму, позначаються як «відносні», тоді як «абсолютне» стосується стану розуму, в якому спантеличеність припинилася і в якому немає навіть найменшого сліду концептуального фокусу, навіть щодо самого неіснування. Як сказано:

\begin{verse}
        Коли поняття реального і нереального \\
        Відсутні перед розумом, \\
        Немає іншої можливості, \\
        Окрім як перебувати в цілковитому спокої, поза концепціями. [3] \\
\end{verse}

\subsection{Природний стан власного розуму}

\subsubsection{Тимчасове розуміння в термінах двох істин}

\subsubsection{Отаточне розуміння, в якому істини є неподільними.}

\section{Вчення про те, як це втілити на практиці}

\subsection{Пряма практика для тих, що із найгострішими здібностямми}

\subsection{Поступова практика для тих, що із менш гострими здібностями}

\end{document}
