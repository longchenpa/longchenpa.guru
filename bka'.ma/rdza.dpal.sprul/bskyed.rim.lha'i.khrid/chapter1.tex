\section*{}
Вклоняюся славному переможцю Ваджрасаттві!

\begin{verse}
Чистий потенціал - абсолютно чистий за природою.
І вільний від завіс,\\
Удинство проявів та порожнечі,\\
Чудові маніфестації Мудрості.
Щоб цей стан трьох Тіл було легко досягнуто,\\
Я поясню те, чим керуються.
У медитації йоги божества.
\end{verse}

Живі істоти блукають в Сансарі, що обманює їх ('khrul ba'i 'khorlo)
внаслідок того, що їх просвітлений потенціал тимчасово нечистий.
Стадія Зародження К'єрім практики йідама встановлює це
стан як просвітлені Тіло, Мова та Ум татхагати, таємне
і незбагненний вимір абсолютного простору. Тут я
обіцяю повністю розкрити цю тему, яка називається "Сходи в Аканіштху".

Текст складається із трьох основних розділів:\\
\\
\begin{tabular}{ll}
1 & Безпомилкова причина - основа вступу в К'єрім. \\
2 & Безпомилкова умова - шлях медитації. \\
3 & Чистий результат - спосіб досягнення \\
& стан єдності.
\end{tabular}

\newpage
\section{Безпомилкова причина.\\Основа вступу \\в Стадію Зародження}

В "Етапах Шляху" (lam rim) говориться:

\begin{verse}
Має скарб,\\
Досконалим потоком та інтересом,\\
Знанням тантри та активності,\\
Сутнісними настановами та теплом,\\
— Такий учитель, наділений вісьмома рисами.
\end{verse}

Тут вказується, що необхідно розпочинати вивчення Стадії Зародження під
керівництвом Ваджрного Вчителя, що має вісім характеристик, і служити йому
(або їй) трьома способами. Наступний етап - дозрівання, за допомогою отримання всіх
необхідних посвят:\\

\begin{tabular}{ll}
1 & Сприятливе зовнішнє посвята (phyi phan pa);\\
2 & Внутрішнє посвячення сили (nang nus pa); \\
3 & Глибока таємна посвята (gsan ba zab mo).
\end{tabular}\\

З цього моменту необхідно підтримувати різні обітниці самі та зобов'язання
(Dam tshig dang sdom pa). Також слід займатися практикою Стадії Зародження йоги.
божества.\\

\begin{siderules}
Ця частина містить два розділи:
1) вчення про очищення звичних тенденцій, пов'язаних з чотирма видами народження та
2) трьох Самадхі - основу для процесу зародження. \ tablefootnote {
Тут і далі виділені абзаци тексту відповідають коментарю Патрула Чок'ї Вангпо,
який називається "Роз'яснення важких місць у Стадії зародження та Йозі божества"
(bskyed rim lha'i khrid kyi dka' gnad cung zad bshad pa bzhugs).}
\end{siderules}

\subsection{Чотири способи народження та зародження}

\begin{siderules}
Вцілому, ключовим моментом (смислом) всіх шляхів Великої колісниці є очищення
природи Сансари - Істини страждання та її джерела, а також привнесення
результуючого стану Будди на шлях (lam du byed ра). І хоча Істина страждання
проявляється у різних формах, всі вони вкорінені у народженні (skye ba).
Далі всі форми страждання можуть бути розділені на народження і смерть.
Тому є дві стадії, які очищають цей подвійний процес народження та смерті:
Стадія Зародження та Стадія Завершення. Вся текстова традиція Таємної мантри
пов'язана із цими двома практиками.
\end{siderules}

В "Славній магічній мережі" (dpal sgyu 'phrul drwa bа) сказано:

\begin{verse}
Щоб очистити чотири способи народження,\\
Є чотири способи зародження:\\
Дуже складний та складний,\\
Простий та дуже простий.
\end{verse}

\begin{siderules}
Як тут показано, є чотири типи народження. Ці чотири, у свою чергу, пов'язані з
чотирма типами практики Стадії Зародження: короткої, середньої, розлогої та дуже
розлогою (bsdus 'brin rgyas pa dang rgyas pa).
\end{siderules}

\subsubsection{Короткий опис граничної простоти}

Ті, хто мають найвищу проникливість (вищі здібності) (dbang ро
yang rab), практикують ритуал Стадії Зародження граничної простоти (shin tu spros pa med
pa) через розвиток внутрішньої енергії розуму (rtsal sems), який пов'язаний із загальним
думкою (lta ba spyi) найвищої царської природної колісниці (rang bzhin theg mchog
rgyal ро). І потужний імпульс (btsan thabs), що виникає з цього процесу, дозволяє
вправлятися у нероздільності Стадій Зародження та Завершення. Не спираючись на слова,
природа розуму постає у своєму первісному стані, як досконала форма божества.
Це відбувається подібно до миттєвої появи відображення в дзеркалі. Наступна цитата
описує цей підхід.

\begin{verse}
Божество - ти, і ти - божество.
Ти і божество виникає разом.\\
Оскільки Самайя і Мудрість недвоє,\\
Немає потреби не запрошувати божество,\\
Не просити його сісти.
Емануючий із себе, самоприсвячуючий,\\
І самоусвідомлююче це Три Корені.
\end{verse}

У цій формі Стадії Зародження природа божества невід'ємна та досконала
образом присутній у ілюзорному прояві Мудрості (yeshes sgyu mа).
Це очищає "чудесне народження (rdzus skyes)". Пояснюючи далі, всезнаючий Лонгченпа пише:

\begin{verse}
Також як чудове народження виникає миттєво,\\
Немає необхідності починати із самого початку,\\
А потім медитувати
На Стадіях Зародження та Завершення.\\
\end{verse}

\subsubsection{Проміжний середній простий підхід}

Істина недвоїсності Мудрості та Простору дозволяє тим, хто володіє
найвищою проникливістю (здатністю) (dbang ро rab tu), практикувати просту
Стадію зародження, використовуючи миттєвий підхід (cig char du 'jug pa). У такій практиці
божества стають досконалими в результаті пригадуванняїх сутності (snying ро dran
ра), і вони не зароджуються у вигляді слів. Можна також практикувати поступове
входження (rim gyis 'jug), при якому божества природним чином виявляються в
відкритій сфері Самантабхадрі, у великій порожнечі Простір мудрості (shes rab kyi
dbyings). Таким чином, набувають єдиної та неподвійної ілюзорної форми
божества. У "Тантрі природного виникнення рігпа" (rang shar) про це йдеться так:

\begin{verse}
Що є миттєва практика?
Божество не зароджують, \\
\indent але воно стає досконалим\\
У момент спогаду сутності.
Як практикують поступове входження?
За допомогою поступового входження\
У Простір та Мудрість.\\
\end{verse}

Як тут зазначено, опора — чудовий палац (gzhal yas khang) і «що спирається»
божественна природа (lha'i rang bzhin), що виявляється досконалим
чином, або за допомогою виголошення їх (божеств) імен, або за допомогою
пригадування їхньої сутності. Такий вид стадії зародження очищає звичні тенденції.
(bag chags), пов'язані з народженням з тепла та вологи (drodg sher skyes), а також
передбачає впізнавання нероздільності блаженства і порожнечі (bdes tong).

\subsubsection{Детальний та дуже детальний підходи}

Дуже детальна Стадія Зародження очищає народження з яйця (sgong skye) та
призначена для тих, чий розум схильний до концептуалізації (rtog bcas). У зв'язку з цим
причинний херука виникає як результуючий Ваджрадхара (rdo rje 'dzin ра), коли
медитують на різних аспектах дитини-себе та дитини-інших. У навчаннях розділу
"Садхан Великих Восьми [Каг'є]" (sgrub sde chen ro bka' brgyad) говориться, що існує
п'ять кроків, коли "інших роблять своїм сином" (bdag gi sras su gzhan bya ba): \\
\\
\begin{tabular}{ll}
1 & З причинного насіннєвого складу світло поширюється \\ 
& у поза і втягується назад (spro bsdus), \\ 
& породжуючи первинних Подружжя (yab yum);\\
2 & Будди десяти напрямків притягуються (bkug) \\ 
& і розчиняються (bstim ра) у просторі;\\
3 & Збираються живі істоти та їх завіси (sgrib) очищаються;\\
\end{tabular}

\begin{tabular}{ll}
4 & Проголошується (brjod ра) велич недвійності;\\
5 & ​​Божества витягуються (з'являються?) (bton) \\ 
& з простору і розміщуються (dgod ра) в мандали.\\
\end{tabular}

\begin{siderules}
Якщо коротку та середню стадії легко зрозуміти, то в дуже широкій стадії практики,
описаної вище, ми зустрічаємося з такими темами як "наш власний син" та "син
інших". Наступний уривок розглядає цю тему з погляду розділу тантр
(Rgyud SDE). Щодо "свого власного сина" в "Магічній мережі" сказано:

\begin{verse}
Знаючи, що сам Плід є шляхом,
Природним та без протиріч,\\
Медитируй на всіх мандалах та тиглі\\
Без винятку, як на своєму синові.
\end{verse}

\end{siderules}

"Робити себе сином інших" (gzhan gyi sras su bdag bya ba) складається з восьми частин: \\

\begin{tabular}{ll}
1 & Первинне подружжя розчиняється (bzhu) \\ 
& і перетворюються на причинний насіннєвий склад;\\
2 & Подружжя зароджується з цієї насіннєвої мови;\\
3 & З концепцій (rtog tshogs) чоловіка зароджуються склади;\\
4 & З дружини випромінюється світло і \\ 
& закликає божество (gsolbagdabра);\\
5 & ​​Всі мандали розчиняються в собі як чоловічі \\ 
& і жіноче подружжя, породжуючи гордість божества \\ 
& Мудрості (ye shes par nga rgyal); \\
6 & Подружжя з'єднується (sbyor ba), \\ 
& і просторі породжується мандала;\\
7 & 42 види власних концепцій (rtogра) перетворюються\\ 
& у божеств і розходяться зовні (phyir gdon ра);\\
8 & Божества мудрості запрошуються (spyan drang), \\ 
& запечатуються (rgyas gdab ра) і т.д.\\
\end{tabular}

\begin{siderules}
І у зв'язку з "сином інших" сказано:

\begin{verse}
Якщо уми нероздільно залучені та злиті\\
Одним, зверненим з благанням до іншого,\\
Тоді який сенс ставати сином?
\end{verse}

Тоді як у "Тантрі кіли" (yang phurpa'i rgyud) щодо "сина інших" сказано:

\begin{verse}
Подібно до сили Будд,\\
У ритуалі спонтанності\\
І на завершення гілок ритуалу,\\
З'являється мудрий – син Будд.
\end{verse}

А також щодо "власного сина" сказано:

\begin{verse}
Син і цариця виникають.
З явищ Сансари.
\end{verse}

Тоді як у розділі "Восьми великих садхан" сказано:

\begin{verse}
«Є п'ять способів зробити інших своєю дитиною».
\end{verse}

І хоча ця тема досить ясна у кожному окремому тексті, схоже, що немає ясного
пояснення, в якому було б точно вказано, що означає "власний син" та "син
інших". [Саме] Тому дана тема може здаватися важкою для пояснення.
Однак, коли ця тема досліджується, перший та останній уривок можуть бути зрозумілі як учні
тому, що основа зародження (bskyed gzhi) є Дхармат самості - Сугатагарбхой.
Таким чином, візуалізуючи форму цього причинного Власника Ваджри (Ваджрадхари),
"інші" - всі чіпляння та прояви (snang zheri), які ми сприймаємо в Сансарі та
Нірвані разом із проміжним станом (bar srid) — збираються, а потім
"роблять сина", витягуючи його та лиха. Цей процес називається "робити інших
своїм власним сином".
\\
Коли "роблять себе сином інших", то всі прояви та сприйняття Сансари та Нірвани
("Інші") візуалізуються як головне божество мандали (Ваджрадхара). Це перетворює
всі накопичені чіпляння по відношенню до скандхів, дхату та аятанів у насіннєвий склад,
який потім стає "зробленим сином" внаслідок вилучення з лона. Таким
чином стають "сином інших".
\\
У середньому уривку сказано, що сансаричні істоти є повністю і досконало
просвітленими, як Будди трьох часів. І тут живі істоти розглядаються як
"самі", а Будди як "інші". Роблячи себе "дитиною Будд", з'являються в результаті
зародження себе як головного божества мандали під час практик удосконалення,
таких як ритуал спонтанності (grub pa'i choga) та п'ять гілок ритуалу (cho ga'i yan lag lnga).\\
\\
"Робити інших своєю власною дитиною", з іншого боку, відноситься до процесу,
в якому все виявляється як збори божеств у колі мандали є "іншим",
а всі сансаричні істоти - "собою". Послідовність, у якій кожен із них виникає,
гармонізується (sgo bstun) у цьому підході, що веде до союзу чоловічого та жіночого
подружжя, а також до подальшого розвитку через витяг з лона. Перші два з цих
пояснень представляють цю концепцію з погляду основи зародження, тоді як
наступне — з погляду процесу зародження (bskyed tshul).\\
\\
Тут може виникнути питання: чи ці два підходи уявлення себе синами
хоч трохи різними. Однак, у "Колесниці всезнаючих" (rnam mkhyen shing
rta) є пояснення, яке пов'язує усі три. Тому я закликаю розумних істот
ретельно досліджувати це пояснення і відкрити браму до цього ясного пояснення.
\end{siderules}

Відповідно до тексту "Велика чарівна мережа" (rgyu 'phrul drwaba chenро) починати
необхідно з прийняття Притулку і зародження Бодхічітти. Те, що відбувається далі і пов'язане
з народженням із яйця, здійснюється у два етапи. Спочатку, миттєво уявіть себе
початковим подружжям (yab yum), а потім запросіть у простір перед собою мандалу,
на яку ви медитуєте. Далі звершуйте підношення, вихваляння, молитви,
постійні каяття (mchod bstod gsol gdab rgyun bshags) і т.д. Після цього використовуйте
ВАДЖРА МУ, щоб знову перебувати в медитативному рівновазі (mnyam par gzhag)
сфері Порожнечі. Цей останній пункт характерний виключно для цього підходу.
Знаючи це, наступне уявлення може бути використане як у гранично-розгорнутому
підході, так і в середньо деталізованій Стадії Зародження, яка очищає народження
з утроби.\\
\\
У таких випадках основний наголос робиться на майстерних методах Стадії Зародження.
Також як буйствующий слон може бути керований гаком або натовпом, є два підходи до цієї
активності. Форма божественної обителі та самих божеств очищають два елементи: опору та
те, що спирається. Перше - [опора] відноситься до зовнішнього всесвіту, а друге
[те, що спирається] - до фізичного тіла. Існування їх обох
[всесвіт-«опора» і ми — істоти-«що спирається»] спирається на різні
звичні тенденції (bag chags). Наступний метод заспокоює схильність розуму
випускати зовнішні об'єкти (yul la 'phro ba) і полягає у встановленні
стовпа (rtod ра) глибокого медитативного занурення (ting ge 'dzin).
\\
Цей підхід дозволяє поєднатися з сутністю очищення, вдосконалення та
дозрівання (dag rdzogs smin). А точніше, оскільки це відповідає характеристикам
Сансари, існування (srid ра) очищується та покращується. І оскільки це подібно до шляху
Нірвани, результат досконалий в основі (gzhi la rdzogs). І, нарешті, обидва вони призводять до
дозрівання Стадії Завершення. Це загальне розуміння має надзвичайно важливе значення.

\begin{siderules}
Великий спосіб зародження передбачає зародження п'яти маніфестацій просвітлення
(Mngon byang). У цьому контексті робиться потрійний поділ: 1) маніфестація
просвітлення основи; 2) маніфестація просвітлення колії та 3) маніфестація просвітлення
Плід. Перше пов'язане з основою очищення, сутнісною природою (rang bzhin ngo bo)
Сансари. Друге — з процесом очищення, де докладають зусиль на шляху та зароджують це в
своєму потоці [розуму]. Третє це результат очищення, стан актуалізації результату,
яке стає досконалим (mthar thug). І хоча є три поділи, насправді це
ні що інше як Ясне світло ('od gsal) чотирьох пустот і супроводжуюча їх єдність
(Zung 'jug). Як такі, вони охоплюють головні моменти всіх шляхів Таємної Мантри.
\\
Звичайні тіло, мова і розум, разом зі зборами (tshogs), або білий і червоний елементи, а
також прани - розум (rlung sems) разом із зборами, є виворально просвітленими
і мають сутність п'яти мудростей (ye nas ye shes lnga'i ngo bor' sangs rgyas pa). Така
маніфестація просвітлення на етапі основи.
\\
Символічна Мудрість (dpe'i yeshes) пов'язана з шляхом накопичення і шляхом з'єднання,
тоді як актуалізація істинного Ясного світла.
передбачає чотири порожнечі, що супроводжуються єдністю (союзом), які
виникають на шляху бачення та шляху медитації. Ці фактори зароджуються у нашому потоці
(rgyud) через Стадії Зародження та Завершення. На початку медитують на п'ятимовах — місяць,
сонце і т.д. Потім підключаються п'ять мудростей, пов'язаних із проникненням ключових
моментів (gnad du bsnan) білої та червоної сутностей, а також вітру та розуму. Цей процес
передбачає маніфестацію просвітлення по дорозі.\\
\\
Коли ж досягають остаточного результату цього процесу, то білий елемент
природним чином є як ваджрне тіло, червоний елемент — як ваджрна
мова, а свідома свідомість (shes rig) - як ваджрний розум. Таким чином, виявляються як
втілення просвітленого розуму - Мудрість Дхармакаї, Ясне світло досконалого
результату. І Дхармакая, своєю чергою, не окрема від Самбхогакаї, яка втілює
океан знаків, рис та досконалих якостей (mtshan dpe yon tan rgya mtsho). Така єдність
за межами навчання (mi slob ра 'zung 'jug), маніфестація просвітлення досконалого
Плода.
\\
Основа очищення в цьому процесі - це дуже тонка біла та червона сутності, разом з
вітромумом. Коли вони перебувають у нечистому стані, вони утворюють основу нашого
звичайного тіла, мови та розуму. Це також тонкі фактори, які затемнюють три прояви
(snang gsum gyi sgrib), а також перешкоджають досконалості істинного Ясного світла, як є
(Ji bzhin). Тільки шлях Ваджраяни вважається протиотрутою для цих факторів. Ось чому
кажуть, що єдиний шлях — це Тайна Мантра.
Коли здійснено і повністю зрозуміло зв'язок між основою очищення, процесом
очищення та результатом очищення, досягають знання всіх ключових моментів шляху
Ваджраяни.
\end{siderules}

\subsection{Три самадхи}

На початку практика Стадії Зародження проходить через [етапи] трьох самадхи:\\

\begin{tabular}{ll}
1 & самадхи Такості;\\
2 & самадхи Ясного Світла;\\
3 & причинний самадхи.
\end{tabular}\\

Далі описується сутнісна природа самадхі такості
gyi ngo bo): Усередині і поза собою, розум залежить від будь-якої основи. Немає кореня, з якого
він зростає. Він не існує в якійсь онтологічній крайності (dngos ро 'mtha'). Він
не чоловічого, не жіночого та не середнього роду. У нього немає кольору, структури чи форми.
Однак, оскільки він за своєю природою є Ясним Світлом (rang bzhin gyi 'od gsal), він
також не є простим ніщо. В результаті спогади Мудрості Дхармати, як
вона є насправді (chos nyid ji ltar ba'i ye shes), існування смерті очищається в
Дхармакайю. Впевненість у тому, що речі постійні, очищується так само, як і сфера без
форми (gzugs med khams). Таким чином, це називають "самадхі таковості".
Самадхи Ясного Світла ('od gsal) називається так, оскільки природне сяйво цього
великого порожнього Світла є зрівнює і об'єднує співчуття за
по відношенню до всіх живих істот, а також воно звільняє від нігілістичних
поглядів та сферу форм (gzugs kyi khams). Крім того, незбагненна виявляється
енергія (rol rtsal) цієї поширюючої Мудрості (mched ра'i ye shes) готує грунт для
перетворення проміжного стану бардо на Самбхогакайю. Це так зване "все
що висвітлює (kun tu snang ba) самадхи".\\
\\
З цього саме усвідомлення (rigра nyid) з'являється потім у формі складів А, ХУМ або
ХРІ. Цей процес відомий як "причинний метод", а також "корінний метод". Причинне
самадхі очищає свідомість нинішнього моменту існування, яке готове увійти до
нове місце перебування. Крім того, воно очищає сферу бажань ('dod khams) і призводить до
дозрівання народження в Нірманакаї. Таке причинне самадхи.

\subsection{Зародження підтримуючої \\ та підтримуваної мандал}

Заклавши основу для Стадії Зародження цими трьома видами самадхи, далі можна
приступати до зародження підтримуючого та підтримуваного. Точна природа цього
процесу викладено в великих зборах тантр традиції ранніх перекладів Ньінгма, у зв'язку з
з тріадою Зародження, Завершення та Великого Завершення, у такому як "Велика таємна
сутність" (dpal gsang bа'i snying po). У цьому підході причинний або корінний метод
призводить до основного способу візуалізації палацу божества та трону. І це, в
своєю чергою, веде до неймовірного (bsamyas) способу медитації на всій
підтримуваної мандалі.
\\
Крім того, медитатація на формі чудового палацу у безкрайньому просторі
благословляє (byin brlab) нечисту природу судини (тобто цього світу) як Аканіштху.
Тридцять сім факторів Просвітлення (byang chub kyi chos sum curtsa bdun) - це те, що
встановлює сутнісний зв'язок (sbyor bа'i ngo bo) між основою, шляхом, Плодом,
результатом та чистотою. Природа зв'язку пояснюється в дев'ятому розділі тексту
"Велика колісниця (shing rta chen mо)".\\
\\
Тут, однак, ми зосередимося в основному на стадіях візуалізації підтримуваного
божества (brten ра), пов'язуючи спосіб, яким виникає ваджрне тіло, зі Стадією
Зародження. Цей процес очищає основу (об'єкт) очищення (sbyang gzhi).
Втантре "Херука Галпо" (he ru kа gal ро) сказано:

\begin{verse}
Перше — Шуньята та Бодхічитта,\\
Друге - виникнення Семени, \\
Третє — досконала форма
Четверте - встановлення [корінного] Слогу.
\end{verse}

Тут йдеться про те, що смерть та проміжний стан очищаються Шуньятою.
(пустотністю) і Бодхічиттою. Свідомість у формі безтілесного духу (driza), готова увійти
у з'єднання насіння та яйцеклітини, очищається збиранням насіння (sa bon bsdu ba). Потім
поступово розвивається тіло, яке згущується десятьма вітрами (енергіями) (гlung bcu).
Цей процес очищається досконалою формою (gzugs rdzogs ра). Будучи народженими,
чуттєві здібності (органи сприйняття) (dbang ро) проявляють активність (sad ра) по
по відношенню до своїх об'єктів. І це, своєю чергою, очищується встановленням складу (yi ge
'god ра) і т.д. Це пояснення має відношення до символів чотирьох маніфестацій.
просвітлення (mngon byang bzhi' brda).
\\
Однак тут буде дано пояснення з погляду поступового розвитку, що
відповідає загальному погляду "Зборів Тантр (rgyud sde)". Згідно з батьківськими тантрами
зародження відбувається за допомогою ритуалу трьох важдр, тоді як згідно
материнським тантра це відбувається за допомогою п'яти маніфестацій просвітлення. У
обох цих традиціях основа очищення пов'язана з чистотою процесу очищення.

\begin{siderules}
Другий розділ кинутий до трьох самадхи, підходу, який є особливим для традиції.
ранніх перекладів школи Нінгма. В інших системах нічого не відбувається, крім
зародження проміжного стану (bar srid bskyed), коли вже було зібрано накопичення
Мудрості. Але в цій традиції говориться, що сутність (snying ро) порожнечі - це
співчуття, і що імпульс, що походить із союзу, призводить до виникнення чудової
еманації (sprul pa'i lha) з тілом, ликом та руками, а також до здійснення блага інших
за допомогою чотирьох видів просвітленої активності. Такий безпомилковий шлях Великої
колісниці, з'єднання двох накопичень (tshogs).
\\
У межах нечистого сансаричного існування неможливо увійти до утроби, якщо
бажання і прихильність (sred len) не підтримують карму, яка підштовхує до
майбутньому народженню. І якщо ця карма підтримується, тоді відбувається зачаття. Таким же
чином, коли підтримуються імпульсом (phen pas gsos btab) великого співчуття, тоді
зароджують мандалу божеств, що виникає зі стану сяючої порожнечі. Це
відповідає об'єкту очищення у межах Сансари. І це також відповідає тому, що
відбувається у рамках результату завершення цього процесу. У цей момент
неконцептуальне співчуття втілюється з Дхармакаї та працює на благо нескінченного
кількості живих істот.
\\
Поки не втрачають на увазі еліксир (rtsi) порожнечі та співчуття, Стадія Зародження не
може призвести до зміцнення нормального стану. Медитуючи відповідно до шляхом
остаточної досконалості (nges rdzogs) досягають стійкості в самадхі таковості.
Завдяки цій стійкості зародження та завершення перебувають у союзі, а також у потоці
розуму виникає справжня Стадія Завершення, яку Буддаачарья Джнянапада описує
як неподвійність глибини та ясності (zab gsal gnyis med). Джнянапада підтримує
медитацію, у якій Стадія Зародження запечатана Стадією Завершення. І хоча сказано,
що ранні вчителі не розуміли цього моменту, цей погляд відповідає особливому
способу представлення цієї теми в ранніх традиціях школи Нінгма.
\\
Коли медитують на трьох самадхи у відповідність з шляхом остаточної досконалості, ті,
хто отримав посвячення і зберігає обітниці сама, на початку має бути введений (ngo sprod) в
справжню природу думки. Наступний ступінь - це здобуття впевненості (yid ches) в
цьому погляд і за допомогою опори на логічні міркування (gtan tshig). Потім
необхідно помістити безтурботний і природний розум у стані невигадливої ​​простоти
(mа bcos spros bral) за допомогою одного з двох методів: або за допомогою перебування
після прозріння (mthong ba'i rjes la 'jog pa), або через перебування в
безпосередності повного усвідомлення (rigра spyi blug su 'jog pa). Це дозволяє припинити
потік спогадів і думок, що переміщаються, а також звільнитися від помилок млявості і
сум'яття (bying rgod). Знайомство з цим станом вводить вітри розуму до центрального каналу.
Видимі прояви десяти знаків (rtags bcu) досягають стану досконалості і,
найкращому разі, можуть навіть зароджувати символічну Мудрість, яка спирається на три
прояви.
\\
І навіть якщо це не так, можна тренувати розум, зміцнюючи розуміння, що порожнеча
вільна від концептуальних вигадок (spros ра). Потім, коли не втрачають на увазі це
розуміння, то бачать, що Сансара проявляється, але не існує. Тоді з'являється
можливість розвивати подібні ілюзії та вільне від фіксації співчуття, яке
пронизує весь простір. Завдяки знайомству з цим, коли розум фіксується (blо
bzhag) на порожнечі, без зусиль виникає відчуття співчуття до всіх істот,
які не осягають цього і які обманюються насправді неіснуючими
проявами. І навіть коли медитують на співчутті, не буде зроблено помилку
конкретизації (а 'thas) себе та інших. Натомість з'являється впевненість у порожнечі,
тому факті, що прояви виникають у взаємній залежності та ілюзорні, а також у тому,
що їх справжня природа не встановлена.
\\
Знайомство (nges shes) з пустотою, яка має своєю сутністю співчуття - це
Ключовий момент перших двох самадхи - Такості та Ясного світла. З цієї причини
відсутність одного з них не дозволить вам досягти Досконалого Просвітління (rdzogs pa'i
byang chub). Тому не варто навіть говорити, що буде, якщо не буде обох. Як
писав Сараха:

\begin{verse}
Без співчуття думка порожнечі -\\
Немає досягнення найвищого Шляху.\\
Якщо ви медитуєте тільки на співчутті,\\
Як ви звільнитеся від Сансари?
\end{verse}

Ці два фактори є тим, що перетворює Стадію Зародження на шлях Великої
колісниці. І навпаки, будь-яка Стадія Зародження без цих двох [самадхі] нічим не
відрізняється від не-Буддійської [практики].\\
\\
Ця порожнеча зі співчуттям як сутність (snying ро) проявляється як А, ХУМ,
ХРІ та інші причинні насінні склади (rgyu'i sa bon). Коли ви медитуєте на цих
складах і знайомтеся з ними, то можна перебувати в медитативному стані (mnyam par
bzhag) так довго, як хочеться. І коли можуть це робити, то можуть еманувати незліченне
кількість складів, отже, вони заповнюють весь простір. Потім знову збирають їх у
первинному складі і поринають у медитативний стан. Необхідно продовжувати
вправлятися у цих практиках, доки досягнуто вісім вимірів ясності і
стійкості (gsal brtan gyi tshad brgyad). Цей процес відомий як "тренування тонких
складів (phra ba yig'bru la bslab ра)". Знайомство тільки з цим процесом дозволяє досягти
всіх духовних досягнень (dngo sgrub) та знаків реалізації (grub rtags).
Якщо немає можливості медитувати на трьох самадхи у відповідність до Шляху певного
досконалості (lam nges rdzogs), то можна замість цього використовувати підхід
медитації устремління (mos sgom). У такому разі на початку кожної практики мають бути
здійснено різні аспекти ритуалу. Не задовольняйтесь простою вимовою
слів! Натомість розслабтеся зсередини і перебувайте в медитативному стані до
реалізації самадхи Таковості Ясного світла. Потім поступово медитуйте на причинному
самадхи та інших стадіях практики. Поки що практика Стадії Зародження не вкорениться у
первинно в елементі практики (nyams len gyi gtso bo) — порожнечі та співчутті — не
слід віддавати пріоритет рецитації, випромінювання та втягування (bzlas ра dang spro bsdus),
а також іншим подібним факторам.
\\
Під час практики Стадії Зародження, чи медитуєте ви та використовуючи підхід медитації
устремління чи певної досконалості, необхідно момедитувати на трьох самадхи,
неокремих один від одного. Але недостатньо практикувати їх іноді. Наприклад,
коли на стіні малюють зображення, то стіна є основою для малювання, оскільки, якщо
її немає, тоді нема на чому малювати. І навіть якщо стіна є, але вона не від штукатурена, на ній
також не можна нічого намалювати. Штукатурка в даному випадку сприяє
умовою (lhan cig byed pa'i rkyen) зображення. Коли ж зображення намальовано, стіна,
штукатурк а і саме зображення, як і раніше, залишаються. Так само, навіть коли
медитують на всьому колі мандали, що виникає з причинного самадхи, як
божества проявляється саме порожнеча та співчуття. Тому ці три самадхи мають
практикуватися в єдності.
\\
Але якщо звичайні прояви ще не очистили до порожнечі, то, як ви можете медитувати
на колі мандали? Той факт, що всі явища є порожнечею, і що Сансара і
Нірвана нероздільні, є причина, через яку ми можемо це актуалізувати завдяки
медитації на колі мандали. Іншими словами, порожнеча – це основа для Стадії Зародження.
Як сказано:

\begin{verse}
Для кого порожнеча можлива,\\
Для того все можливе.
\end{verse}

Якби всі явища були порожніми, а звичайні прояви були істинно встановлені
(Bden par grub), тоді медитація Стадії Зародження була б неможлива. І, як
вказується у наступній цитаті:

\begin{verse}
Можна присвятити пшоно в рис,
Але рис все одно не з'явиться.
\end{verse}

Навіть якщо всі явища таким чином осягаються як порожні, без великого імпульсу
співчуття ви не зможете втілити рукаї, щоб допомогти іншим. Це подібно
шравакам і пратьєкабуддам, які входять у стан припинення ('gags ра) і не
допомагають іншим еманаціями рупака.\\
\\
Коли розуміють цей момент, то це подібно до наступного висловлювання: "Усі ці явища
подібні ілюзії, а народження подібно до прогулянки парком..." Інакше кажучи, більше не
перебувають у існуванні, тоді як внаслідок співчуття нечіпляються за спокій (zhi).
Такий великий Шлях синів Переможного. З цих причин, розуміння того, що три
самадхи неізольовані один від одного, це найважливіший момент.
\\
Сьогодні більшість із тих, хто має деяке переживання Стадії Зародження,
зберігають злість на своєму смертному одрі. У результаті вони перероджуються у сфері Владики
Смерті (Gshin rje). І це відхилення призводить до того, що вони стають демонами.
завдають шкоди живим істотам. Причина цього полягає в тому, що ці люди не освоїли.
goms ра) порожнеча з співчуттям як свою сутність. Є безліч людей,
які мають інтелектуальне розуміння порожнечі, можуть ясно візуалізувати на
Стадії Зародження, а також повністю мантри, що завершили повторення, але які тим не менше
менш закінчують тим, що перероджуються як демони. З іншого боку, ви ніколи не
побачите злобного вовкулака, який би володів еліксиром великого співчуття.
\end{siderules}

\subsubsection{Ритуал трьох ваджр}

Символічні атрибути, такі як п'ятикутна ваджра, очищають розум. Ясно медитуючи
на їх формі, основа (об'єкт) очищення - розум, очищається [і перетворюється] на ваджрний розум.
І подібним чином, основа очищення — звичайна мова, вдосконалюється у ваджрну мову.
Процес очищення у разі полягає у перетворенні цих символічних предметів
(атрибутів) (phyag mtshan) або прикраса (mtshan ра) їх у медитації складом ХУМ та
іншими. Наступна стадія очищення передбачає ритуал випромінювання та поглинання (spro
bsdus) променів світла, що приносить подвійну користь, а також подальше перетворення на
форму божества, наділену всіма прикрасами та шатами. Ця медитація розвиває
тіло, що є об'єктом очищення, до ваджрного тіла. Цей етап утворює
ритуал трьох ваджр.
\\
І ці три [ваджри] також можуть бути застосовні до процесу народження з утроби. Перша
стадія очищає злиття червоного та білого елементів, а також подальше входження
свідомості у проміжний стан. Друга стадія очищає п'ятичастинний процес розвитку
зародка в утробі, що складається з [формування] овальної форми та інших стадій,
які йдуть за сходженням насіння, крові (яйця) та свідомості. Третя стадія очищає
народження, яке виникає, коли з'єднуються насамперед розсіяні елементи і повністю
формуються тіло та органи чуття.\\
\\
Цей процес також утворює стадію того, що відбувається після отримання Плоду.
Взаємозалежність (rten 'brеl) створюється певними діями, які
маніфестуються татхагатами. Особливо це стосується просвітленої активності
входження до утроби ваджрної цариці, народження і т.д., яка демонструється Буддами,
коли вони набувають форми нірманакаї, щоб підкорювати істот необхідними методами.
\subsubsection{П'ять актуалізацій Просвітлення}

У цьому обговоренні ми поєднуємо п'ять зовнішніх актуалізацій Просвітлення (phyi' mngon byang)
йога-тантри (rnal' bug rgyud) з п'ятьма внутрішніми актуалізаціями
Просвітлення, що описуються в традиції маха-йоги (rnal 'bуог chen ро), поєднуючи причину зі
способом досягнення плоду. У тантрі "Простору ясності (klong gsal)" сказано:

\begin{verse}
Зачаття пов'язане з п'ятьма аспектами Просвітлення,\\
Десять місяців зі способом розгортання десятиу рівнів,\\
І народження - з природною нірманакаєю;
Так, природно, втілюються три Тіла.
\end{verse}

Як тут сказано, період, що починається з проміжного стану та
що триває до моменту, коли шукають фізичне тіло, пов'язаний з шляхом накопичення
(tshogs lam), тоді як дійсне набуття фізичної форми в утробі пов'язане з шляхом
з'єднання (sbyor lam). Період з моменту зачаття і далі відзначається як шлях медитації
(Sgomgyi lam). По суті, десять місяців є десять рівнів (sa bcu),
які ведуть до здійснення природної нірманакаї, нероздільного стану без
тренування (mi slob pa'i zung' jug rang bzhin sprul pa'i sku).
Тут ми пов'язуємо певний процес очищення, вдосконалення та
дозрівання з різними стадіями чотирьох видів народження.
\\
1) Перша маніфестація просвітлення (mngon byang) – це місячний диск. Вона
очищає та вимиває такі об'єкти очищення: сукупність форми (gzugs kyi phung)
ро), елемент простору (nаm mkha'i khams), тіло та свідомість - основу всього (lus kun gzhi'i
rnam shes), затьмарення-клешу незнання (nyon mongs ра gti mug), чоловічий елемент (насіння,
pha'i khams), пов'язаний з народженням з утроби, вологу, пов'язану з народженням з тепла і
вологи, чоловіче (pha), пов'язане з народженням з яйця, аспект порожнечі (stong pa'i cha),
пов'язаний із чудовим народженням і т.д. У контексті Стадії Завершення (rdzogs rim)
сходять шістнадцять радостей (dga' ba) і перша Мудрість розрізняючого розуміння (so
sor rtogs ра) проявляється як природа вправних засобів (thabs kyi rang bzhin). Іншими
словами, розум удосконалюється у своїй сутності (ngo bo). Щодо аспекту Шляху, то
медитують, що це перетворюється на місячний диск. Що ж до Плоду, то це дозріває в
вигляді тридцятидвох благих знаків повного Просвітлення та актуалізації Мудрості
відображення (mеlong ye shes).
\\
2) Друга маніфестація Просвітлення – це сонячний диск. Він очищає та вимиває
наступні об'єкти очищення: сукупність відчуття (tshor ba), елемент землі (sa),
затьмарення-клешу свідомості(?), жадібність і гордість, червоний жіночий елемент, пов'язаний з
народженням з утроби, жіноче, пов'язане з народженням з яйця, тепло, пов'язане з народженням
з тепла та вологи, аспект ясності (gsal ba), пов'язаний із чудовим народженням тощо. У
процесі Стадії Завершення яйце і насіння повністю очищаються, стаючи за своєю суттю
одночасним виникненням блаженства та порожнечі (bde stong lhan cig skyes). Це
передбачає процес висхідної опори (mas brtan) - природи мудрості. Двадцять
елементів (дхату) (тобто. форма та інші скандхи разом із чотирма елементами)
виникають як двадцять порожнеч (Мудрість базового простору явищ
(Дхармадхату) та інші Мудрості у поєднанні з чотирма незмірними) і є
досконалими як Пробуджений розум. Це друга мудрість. Щодо аспекту Шляху, то
медитують, що це перетворюється на сонячний диск. І це, у свою чергу, призводить до
дозрівання вісімдесяти чудових другорядних ознак, що виникають у
стан Плоду, а також до повного Просвітлення та актуалізації Мудрості рівності
(mnyam nyid kyi ye shes).\\
\\
3) Третя маніфестація Просвітлення – це насіннєвий склад та атрибут руки (sa bon
phyag mtshan). Вона очищає і вимиває таке: входження у проміжний стан
свідомості (bardo) (безтілесний розум, що триває) між насінням і яйцем, сукупність
сприйняття ('du shes), елемент вогню (mе), мова (ngag), ментальна свідомість (yid kyi rnam shes),
затьмарення-клешу бажання ('dod chags), а також проміжний стан свідомості бардо,
яке входить до будь-якого з чотирьох видів народження. Щодо Стадії Завершення, то
насіння та яйце очищаються назад-назад (ldog tu dag ра), після чого кармічні вітри (las
rlung) та зв'язок часу (dus sbyor) припиняються. Так досягають досконалості третьої
Мудрості ваджрної стійкості (bstan pa'i rdo rje). Щодо а спекту шляху, то
медитують на тому, що це перетворюється на насіннєвий склад та атрибут руки. Що ж до
Плода, то це дозріває в природу розуміння чіткості (mа 'dres ра) всіх явищ, а
також призводить до повного Просвітлення Мудрості розрізнення (so sor rtog pa'i ye shes).
\\
4) Четверта маніфестація Просвітлення - це змішання насіннєвого складу
атрибут руки в один смак (гоg cig tu 'dres ра). Це очищає та вимиває такі фактори:
змішання насіння, яйця та свідомості в один смак, сукупність формують факторів ('du
byed), елемента повітря (rlung), п'ять чуттєвих свідомостей, затьмарення-кльош заздрості (phrag
dog), а також змішання свідомості проміжного стану, насіння та яйця при народженні
з черева та з яйця, змішання свідомості, тепла та вологи при народженні з тепла та вологи, а також
змішання лише порожньої ясності (gsal la stong ра) чудово городіння зі свідомістю
проміжного стану. На стадії завершення функціонування насіння, яйця та
вітрів повністю очищується. Сутність Утримувача причинної ваджри (rdo rje 'dzin ра)
удосконалюється, оскільки це четверта Мудрість знання всього різноманіття (snyed
ра), що має ваджрну природу (bdag nyid) і що охоплює все, що пізнається. Що стосується
аспекту шляху, то візуалізують, що сутність цього змішування перетворюється на
світиться сферу тиглі ('od kyi thig le). Коли досягають Плоду, це дозріває і
втілюється у вигляді активностей всіх Будд, яка зливається в єдиному смаку та призводить до
досконалому Просвітленню Мудрості здійснення всього (bya ba grub pa'i ye shes).
\\
5) П'ята маніфестація Просвітлення - це досконала форма, яка
походить від цього процесу. Вона очищає та вимиває наступні фактори: завершення
періоду вагітності, сукупність свідомості, елемент води (chu), аспект чіпляння за
Реальність восьми зборів (chos nyid kyi tshogs brgyad), ментальна свідомість, затьмарення
клешу гніву (zhes dang), а також стан, в якому тілесні органи почуттів,
відповідні одному з чотирьох способів народження, що повністю розвиваються. Це
очищується і стає природною нірманакаєю. Спосіб, яким це розвивається та
дозріває на Стадії Завершення, наступний: Насіння, яйце і тонкі прана-розум стають
своєю абсолютно чистою сутністю (rnam par dag pa'i ngo bo). До того ж, результуючий
тримач ваджри та п'ята Мудрість (?), втілення Татхагат, стають нашою сутністю
(bdag nyid). Наступне стосується результуючого стану повного звільнення
(rnam par grol ba), природної Мудрості Будди, яка осягає явища такими, якими
вони є насправді (ji lta ba). На цьому шляху медитують на досконалій формі божества, а
також на його прикрасах та одязі. Коли досягають Плоду, то дозріває нероздільна
природа Мудрості-простору (ye shes dbyings), яка вільна від будь-яких перешкод,
що веде до повного Просвітління Мудрості Дхармадхату (dbying skyi ye shes). У тантрі
сказано:

\begin{verse}
Місяць — це Зерцалоподібна Мудрість,\\
Сім із семи — це Рівність.\\
Розрізняє розуміння описується\\
Як насіннєвий склад та атрибути рук божества.\\
Становлення лише одним — це саме Зусилля,\\
А досконалість – Дхармадхату.
\end{verse}

Хоча є чотири способи медитації Стадії Зародження, коли кожен пов'язаний з
чотирма видами народження, їх необхідно використовувати таким чином, щоб вони
відповідали індивідуальним схильностям (rang rang gi bag chags) та ступеню
знайомства із практикою. Владика Переможних, Лонгченпа, писав: \\

\begin{verse}
\small
Хоча є чотири способи медитації, ти маєш використовувати \\
Той, спрямований на переважаючий спосіб народження. \\
Щоб очистити схильності, медитируй за допомогою всіх.\\
Зокрема, спочатку медитуйте відповідно народженню з яйця,\\
А коли з'явиться стійкість — відповідно до народження з утроби.\\
При більшій стійкості тимедитуй відповідно народженню з тепла і вологи,\\
Коли ж повністю освоїшся, використовуй цю справжню стійкість,\\
Щоб миттєво візуалізувати, відповідно до чудового народження.
\normalsize
\end{verse}