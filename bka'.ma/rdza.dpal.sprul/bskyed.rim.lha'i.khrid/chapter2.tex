\section{Безпомилкова умова}

\subsection{Вправа на Шляху, \\ відповідно до своїх здібностей}

1) Зосередження розуму на божестві (lha la sems bzung ba);
2) Виправлення помилок змін ('gyur ba'i skyon bcos ра);\\
3) Відділення від божества (lha dpral ba);
4) Привнесення божества на шлях (lha lam du slang tshul);
5) Змішування розуму з божеством (lha rang sems dang bsre ba);
6) Об'єднання божества з реальністю (lha don las byar ba);
7) Привнесення переживань шлях (spyod yul lam du khyer ba).\\

\subsubsection{Зосередження розуму на божестві}

Ті, хто мають найвищі розумові здібності (blo chen), повинні очищати
свої здібності за допомогою практики Стадії Зародження без чіпляння та фіксації. При
такому підході прояви божества та його прикрас візуалізуються як повністю
досконалі (yongs sur dzogs ра), ясні та виразні з самого початку. Така форма
великої Мудрості, союзу Зародження та Завершення. Виходячи за межі певної
сутності з конкретною природою, вона [форма-прояв Мудрості] проявляється ясно, хоча
і пуста від сутності (ngo bo). Іншими словами, ясність та порожнеча нероздільні. Подібно
відображенню місяця в озері, її природа виразно проявляється для очей учнів, хоча на
насправді порожня.\\
\\
Ті, хто має середні розумові здібності (blo 'bring), повинні починати свою
медитацію з миттєвого згадування всього прояву тіла божества. Наступна стадія
- Це медитація на ясному прояві голови. Коли її бачення стає стійким,
переходять до медитації на правій руці, лівій руці, на тулубі, на лівій і правій нозі, і,
нарешті, на всій формі божества та на його троні. Тренування Стадії Зародження
ілюзорного та ясного прояву перешкоджає появі погляду нігілізму (chad lta).
Коли втомлюються, необхідно згадувати чистоту і розвивати (очищати) здібності (rtsal
sbyangs) по відношенню до сутності цього процесу у ваджраподібному самадхи. Цей
Ключовий момент (gnad) перешкоджає появі погляду сталості (rtag lta).
Для початківців з малими розумовими здібностями може бути важко
візуалізувати будь-яким із цих способів. Коли не дуже добре знайомі з цим
процесом, необхідно розвивати здібності за допомогою постійного використання
форми (rtag pa'i sku). Візьміть освячений і правильно зроблений образ божества йідама,
зображення або статуетку, виконані майстерним майстром, і помістіть її перед собою.
Без спеціальної медитації розглядайте цю форму з голови до п'ят, не блимаючи. Це
називається "допоміжною практикою приміщення уважності до руху". Спочатку
виникатиме безліч зумовлених думок ('du byed kyi rtog tshogs), що є
переживанням руху (g.yo ba'i nyams), про який говорять, як про "подібну воду,
падаючою зі скелі".\\
\\
Однак, у певний момент це змінюється, коли розгортаються звичні
тенденції, пов'язані із проявом божества. Коли це відбувається, його прояви
виникають чітко, як об'єкти в умі (yid yul du), навіть коли очі закриті. Це
називається "переживанням досягнення (thob pa'i nyams)", яке порівнюється зі злодієм,
який дістає те, що заховано у посудині. Іншими словами, в цей момент насправді не
знають природи божества, як є (ji bzhin).
\\
Подібна візуалізація запечатує фіксацію на звичайних проявах (tha mal gyi
snang zhen la rgyas btab). Ця візуалізація не повинна бути плоскою, як малюнок, або ж
опуклою і твердою, подібно до барельєфу. Це має бути ілюзорна фігура, порожня
прояв, вільний від будь-якої речовинності, без плоті, кісток та нутрощів. можна
розвивати здібності у цьому процесі, розглядаючи приклади без докоркової кришталевої
вази та веселки, які повністю прозорі, сяючі та сяючі (zang thal gsal 'tsher
bkrag mdangs). Дана візуалізація має бути подібна до відображення в дзеркалі.\\
\\
Спочатку візьміть образ, який ви використовуєте як опору для візуалізації (dmigs
rten) і дивіться нею довгий час, медитуючи лише недовго. Коли ваша візуалізація
стане більш ясною, ви поступово зможете скоротити час вдивлення в образ і
продовжити час медитації.\\

\subsection{Виправлення помилок, що приводять до змін}

Існує сім загальних помилок, які можуть виникнути на даному етапі:\\

\begin{tabular}{ll}
1 & відволікання від об'єкта зосередження (dmigs ра brjed ра);\\
2 & заціпеніння або лінь (le lо);\\
3 & страх недосягти мети (ma grub kyi sdogs ра);\\
4 & млявість (bying ba);\\
5 & ​​збудженість (rgod ра);\\
6 & незадоволеність, навіть якщо божество виявилося \\ 
& Виразно (lha gsal bzhin du chog mi shes pa'i rtsol ba); \\
7 & байдужість, навіть якщо візуалізація не виразна \\ 
& (mi gsal kyang btang snyoms su 'jog pa'i mi rtsol ba).\\
\end{tabular}

Також є дванадцять помилок, які призводять до змін у візуалізації:\\

\begin{tabular}{ll}
1 & нечіткість (rab rib);\\
2 & бляклість (mog mog);\\
3 & затемненість (grib ma lta bua'm mun pa);\\
4 & зміни розмірів (sku' tshad 'gyur ba);\\
5 & ​​зміни зовнішнього виду (cha lugs 'gyur ba); \\
6 & зміни обрисів (форми) (dbyibs 'gyur ba);\\
7 & зміни кількості (grangs 'gyur ba);\\
8 & зміни розташування (bzhugs tshul 'gyur pa);\\
9 & прояв лише як кольору (kha dog 'ba' zhig snang ba);\\
10 & прояв лише як обриси (dbyibs 'ba' zhigs nang ba);\\
11 & поступове зникнення (rim par nub pa);\\
12 & неповне прояв (rnam 'gyur ma rdzogs pa).
\end{tabular}

Протиотрути (gnyen ро) від цих помилок такі: Протиотрутою від відволікання
є стійка уважність (dran ра brten). Якщо впадають у заціпеніння, то
розвивають віру та старанність (dad ра dang brtson' grus). Коли перебувають у сумніві, то
зосереджуються на просторі (dbyings), а також стають у світлому і прохолодному
місці, якщо відчувають млявість. Коли виникає збудженість, зароджують жаль (skyo
shas) та спрямовують погляд вниз. Якщо незадоволені, то розслаблюють розум (blo lhod), а якщо
байдужі, то виявляють старанність. Так само, якщо візуалізація нечітка,
невизначена або затемнена, то ставлять перед своїми очима іскристий кристал ('ja')
shel). Потім дивляться на опору для медитації та перевіряють її контури. Після цього її
візуалізують у своєму розумі.
\\
Якщо, з іншого боку, розмір тіла, його прояв, положення чи контури
змінюються, тобто різні речі, які можна зробити. Можна медитувати на тому,
що тіло божества величезне і тверде, що олені граються на його руках, ногах, на пальцях
ноги рук, а також, що голуби влітають і вилітають із його ніздрів. Зокрема, у зв'язку з
мирними божествами відповідні проблеми можуть бути усунені за допомогою
візуалізації цих божеств, як мають природою, має дев'ять чорт (tshul dgu).
Щодо цього в "Страхливому спалаху блискавки (rngam glog)" сказано:

\begin{verse}
Кожна та найчудовіша форма\
Має дев'ять рис.\\
Вони м'які, стрункі,\\
Міцні, гнучкі та юні,\\
Ясні, сяючі, привабливі.
І палаючі інтенсивною присутністю.
\end{verse}

Гнівні образи мають бути як мають дев'ять поз. Як сказано в "Херука галпо тантрі":

\begin{verse}
Чарівна, героїчна та жахлива,\\
Сміється, люта і жахлива,\\
Співчутлива, загрозлива та заспокійлива,\\
Візьми ці дев'ять танцювальних поз.
\end{verse}

Наступний уривок з "Гуру магічної мережі" звертається до проблеми коливань серед безлічі божеств:

\begin{verse}
Навіть численні випромінювання та поглинання ('phro'du) зборів божеств
Є великим чудовим проявом свого власного розуму.
\end{verse}

Як тут стверджується, у цьому процесі ви можете очистити свої здібності,
уявляючи, як форми свити виникають із чудесного вияву одного головного
божества. З іншого боку, якщо з'являється лише колір, потрібно подавати обриси.
Якщо з'являються лише обриси, слід перетворити в різні кольори. Коли
візуалізація сучасно зникає, медитируй, що обличчя і руки дуже тверді (rags ра), і
якщо вона є неповною, то зосереджуйся на візуалізації того, чого не дістає.

\begin{siderules}
Ця тема поділяється на сім категорій, першу з яких легко зрозуміти. Що ж до
другий, то ми знаходимо дванадцять недоліків, які призводять до змін
візуалізації. Перше - це нечіткість, пов'язана з тим, що прояв божества ледве
видно, його контури, обличчя, руки та інші частини невиразні. Наступна — бляклість, це
недостатня барвистість візуалізації. Коли це відбувається, то білий, червоний, синій та
інші кольори візуалізації недостатньо інтенсивні та насичені, подібно до ясного і
блискучому світлу тисячі сяючих сонців. Натомість вони блідо-білі і блідо-червоні.
Також може бути, що фіксуючи свій розум на прояві, що візуалізується
божества, потім ненность перешкоджає баченню розуму. І навіть якщо це не відбувається,
прояв розпливчасто і неясно, подібно до туманного і нечіткого сприйняття, коли
товщина рук і ніг, а також ширина обличчя та очей проявляються невиразно.\\
\\
Інша проблема пов'язана із зміною мрозмірів божества. Ви можете візуалізувати
не що розміром з маленьке насіння, наприклад, але коли виявляється, то це вже має
розмір великого пальця. Це також може бути пов'язано з формою мирного чи гнівного
божества, позбавленого пропорцій, як у поганому малюнку. Коли трапляються зміни
зовнішнього вигляду, то дев'ять поз, прикраси мирних божеств та інші подібні
Показники не відповідають належної візуалізації. Інша проблема виникає, коли
божество змінює форму, наприклад, ви візуалізуєте мирне божество, чиє обличчя має
форму гірчичного насіння, а в результаті його обриси набуває форми привабливого
жіночого божества. Зміна кількості відбувається, коли три особи стають чотирма,
або коли з'являється безліч основних божеств, а чи не лише кілька. Зміна
розташування передбачає зміну позицій лотоса, напівлотоса і т.д.\\
\\
Можливий прояв лише кольору. Іншими словами, є відмінності білого, червоного та інших
квітів, проте товщина та інші характеристики, пов'язані з обрисами візуалізації не
чотки. Крім того може бути прояв лише контуру, а не кольору. Коли відбувається
поступове зникнення, то візуалізація стає все менш ясною, хоча ви і не на
Стадії завершення. І не повний прояв відбувається, коли можуть візуалізувати
кожну частину тіла - голову, руки і т.д., головного і другорядних божеств, але не їх у
загалом.\\
\\
Дев'ять чорт мирних божеств (zhi ba'i tshul dgu) є протиотрути (gnyen) від цих
проблем. Форми божеств, обличчя, руки тощо. повинні бути м'якими (mnyen pa), а не жорсткими,
як кістки чи дерева. Їхні тіла повинні бути пропорційними, що поступово звужуються.
до талії. Їх плоть не повинна бути в'ялою, але пружною. Вони мають бути гнучкими, у тому
сенсі, що їх суглоби та члени рухливі. Їхні тіла мають виглядати здоровими, а їх
шкіра ніжна та юна. Їхні кольори мають бути чіткими та ясними, і вони мають випромінювати світло
у тому сенсі, що вони мають випромінювати безмежне світло. Їх форми мають бути
величні та прикрашені знаками вищих істот, а отже,
привабливими. І, нарешті, у них має бути інтенсивна присутність, яка затьмарює
все своїм блиском.

\\
Гнівні божества також мають дев'ять якостей. Їхній вираз (nyams) пристрасті
має бути полонить (sgegs ра), їх вираження гніву має бути героїчним (dpa' ba),
їх вираження незнання має бути жахливим (mi sdug ра). Такі три вирази
(експресії) тіла. Вони повинні голосно сміятися (gad mо) "ха-ха" та "хі-хі". Вони повинні
люто кричати - "схопи, бий!". Вони повинні грізно ревти, подібно до гуркотів тисячі.
громів та розривів тисяч блискавок. Такими є три їх вислови промови. Також вони повинні
співчутливо брати заблуканих істот та світи під свою опіку. Вони повинні залякувати,
підкорюючи варварів своїм гнівом. І вони мають бути мирними у тому сенсі, що вони
сприймають все, як має єдиний смак дхармат. Такими є дев'ять танцювальних
виразів-експресій.
\end{siderules}

\subsubsection{Відділення від божества}

Третя тема пов'язана з тим, що називається "відділенням від божества". В інтервалах
між цими медитаціями потрібно включати періоди медитації на божестві
неконцептуальним спогляданням (mirtog ра nyam par 'jog ра). Це послаблює відчуття
втоми від практики, що може виникнути. Потім знову медитують лише на
божестві, візуалізуючи ясно і безпомилково, без концепцій.

\subsubsection{Привнесення божества на шлях}

Четверта тема пов'язана з тим, як принести божество на Шлях, або, іншими словами,
як очистити (якщо тиб. sbyang ba, то це-тренувати) свої здібності, коли
стають послідовником цих практик. Візуалізуй божество тим, хто стоїть, сидить
лежачим. Уявляй його, що стоїть на маківці голови, що лежить на спині, обличчям вниз,
що знаходяться далеко, близько, всередині гори, в глибині океану, масивним, як гора Сумеру,
або ж маленьким, як порошинка. Якщо ти можеш медитувати так, тоді, досягнувши
медитативного переживання, подібного до річки, ти привнесеш божество на шлях.\\

\subsubsection{Змішування свого розуму з божеством}

П'ята тема називається "змішуванням свого розуму з божеством". Щоб усунути
будь-які двоїсті фіксації, які можуть бути відносно медитатора та
медитації, вправляйся у нероздільному змішуванні свого розуму з формою божества. Коли
ти зможеш залишатися в повній і досконалій формі божества так довго, як захочеш, а
також не відволікаючись на думки, тоді ти досягнеш стійкого переживання, подібного
горе.\\
\\
І далі настане момент, коли ти зможеш візуалізувати божество дуже
детально, аж до часу на його тілі та зіниць очей. Також, тоді ти не будеш підвладний
ніяким зовнішнім впливам і зможеш медитувати день і ніч на своєму мандалі
божества. На цьому рівні досконалої майстерності твоя здатність стане досконалою.
Є вісім заходів ясності та стійкості, які вказують на досягнення чи не
досягнення цього. Чотири міри ясності - це сяйво (gsal le), чистота (sang nge), блиск (lhag)
ge) та виразність (lhang nge). Чотири міри стійкості (brtan ра) - це нерухомість (mi
g.yo), незмінність (mi' gyur), досконала незмінність (mngon par mi 'gyur) та всемірна
мінливості (cir yang bsgyur). Коли ці вісім заходів ясності та стійкості досягнуто,
виходять на рівень, відомий як "переживання досконалості (mthar phyin)". Тоді
змішують прояви з мандалою божества.

\begin{siderules}
Третю та четверту теми легко зрозуміти. У п'ятій ми бачимо вісімвимірності
стійкість. Перша - це сяйво, яке відноситься до ясного та виразного прояву
того божества, на якому медитують, аж до зіниць його очей. Наступний захід - це
чистота, стан, у якому усвідомлення має сенс життєвої сили (seng bag). Це
ясне, порожнє і виразне, а чи не притуплене і позбавлене ясності сприйняття. Якщо
божество, на якому ви медитуєте, це лише мертвий і твердий прояв, а не схожий
на веселку, як має бути, тоді кожна його деталь має бути охоплена мудрістю
всезнання (thams cad mkhyen pa'i ye shes), аж до пір і волосків на тілі. Коли це є,
то виникають і блискуче присутні сотні якостей, пов'язані з ясністю органів
почуттів, що є блиском. Форма божества також має бути інтенсивно виразною.
Іншими словами, ви не повинні думати про це, виводячи (rjes dpag) присутність і
прояв. Замість цього вони повинні виникати в розумі безпосередньо (mngon sum), з
Виразною ясністю. Такі чотири міри ясності.

\\
Далі йдуть чотири заходи стабільності. Перша - це нерухомість, яка означає,
що медитація не може бути похитнута такими загальними недоліками, як забудькуватість і
ліньки. Друга - це незмінність всупереч недолікам змін візуалізації, таким як
розпливчастість та нечіткість. Коли візуалізація більше не з'являється лише на короткі
періоди часу, але може підтримуватися день і ніч, без відволікання навіть на самі
тонкі думки, вона стає абсолютно незмінною. І, нарешті, коли ви медитуєте на
таких чинниках, як колір божества, його обличчя та руки, рухи, випромінювання та втягування
променів світла, а також все виникає як належить, то практика стає всіляко
мінливою.\\
\\
Втім, здатність зберігати ясність і спокій, коли практикують медитацію на
божестві, - це позитивна якість. Тим не менш, незалежно від того, що вчасно
медитації форма божества є неясною або чітко ясною, хоч і не медитують,
є помилка чіпляння за візуалізацію. Тому коли освоюють ясний прояв
візуалізації, необхідно вправлятися у стадії розчинення (bsdu rim).
\\
Сьогодні, багато людей не практикують стадію розчинення, окрім як у короткому завершенні
медитації. Вони кажуть різне: "Ви ніколи не повинні втрачати з уваги три мандали,
тому не слід робити стадію розчинення і т.д." Це вказує на те, що вони не тільки
не розуміють ключові моменти Стадій Зародження та Завершення, але навіть не бачать
помилковість концептуальної фіксації стосовно вищого божества. Очевидно, що
це ознака незнання. Саме тому вчать стадії розчинення.
Стадія розчинення очищає звичні тенденції, пов'язані зі смертю та встановлюють
причинний зв'язок із Дхармакаєю. На шляху Стадії Завершення ілюзорне тіло розчиняється
у світлі. Стадія розчинення не тільки готує практикуючого до цієї стадії практики, але
також до проникнення ключових моментів (gnad dub snun) ваджрного тіла, а також до
стягування каналів, вітрів і крапель-тиглів у центральний канал. Втім, тільки це і
породжує всі просвітлені якості шляху Ваджраяни.
\\
Чи було розвинене символічне чи істинне Ясне світло у своєму власному стані
буття, є десять знаків, які з'являються, коли елементи послідовно
розчиняються, тоді як з природи стадії розчинення (thim rim) виникає Ясний
світло трьох проявів. Коли він з'являється, всі звичайні помилкові прояви втягуються в
простір (klong), і три прояви розчиняються у Ясному світлі. При цьому є лише
виникнення Дхармати, Мудрості, вільної від вигадок.\\
\\
Коли це відбувається, деякі практикуючі повинні більше приділяти уваги Стадії
Розчини, а не стадії зародження. Це актуально для тих, хто хоче братися до
доброчесним прагненням під час стану глибокого сну та сновидінь, а також для
тих, хто вдень залучений до практики запровадження вітру-розуму в центральний канал. Ті, хто хочуть
практикувати особливу практику Стадії Завершення, яка актуалізує.
Ілюзорне тіло Ясне світло, також повинні зробити упор настадії розчинення. І це ж
повинні зробити ті, хто хоче з'явитися в божественному тілі єдності тренування (slob pa'i
zung 'jug gi lha sku), коли відкидають своє звичайне тіло. Це також стосується тих, хто
прагне до досконалості цього процесу, до актуалізації стану єдності без
тренування. Цей стан досягається завдяки розчиненню в Ясному ментальному світлі.
тіла, що спирається на звичні тенденції незнання. І, нарешті, є ті, хто
закінчує свої дні, так і не досягнувши Просвітлення, а також ті, хто хоче з'явитися в
об'єднаної Самбхогакаї, перебуваючи в Ясному світлі першого проміжного стану,
стані (dgongs ра) Дхармакаї. Ці практикуючі також повинні робити більший наголос
на фазі розчинення, ніж на Стадії Зародження.
\\
Крім того можуть бути і ті, хто лише хоче здійснити нижчі [тиб. smad las] активності,
подібні до підкорення духів і дарування захисту. Для цього вони зосереджуються
виключно на візуалізованих проявах божества йідама, а також на
поширення і збирання променів світла. Такі практикуючі можуть відчути
радість від своїх візуалізацій, а також породити гордість. Однак такий підхід
не відрізняється від не-буддійського. У результаті вони переживатимуть наслідки своїх
негативних дій.
\\
Деякі можуть заперечити, що "це може бути і так, але для чого змішувати нашу
термінологію з термінологією нових шкіл, використовуючи терміни "ілюзорне тіло" та
"Ясне світло". — Який сором казати такі речі! Це послідовники Нінгма, які навіть
небачили певні класи тантр, такі як "Тантра досконалої таємниці (gsang ba yongs
rdzogs)" і "Херука Галпо-тантра". Такі люди нічого не знають про основні тексти,
пов'язаних із практиками трьох внутрішніх тантр!\\

\\
Тим не менш, вважається, що характерна риса (khyad chos) школи Нінгма Ранніх
Перекладів (snga' 'gyur), це твердження, що цитадель розуміння захоплюється в
просторі ваджрної вершини, на піку всіх колісниць (theg rtse rdo rjer tsemo'i dbyings su
dgongs pa). У такому твердженні мається на увазі те, що необхідно відкинути все
форми ментальної діяльності, навіть Стадію Зародження та інші десять природ. Іншими
словами, необхідно перебувати в оголеній відкритості порожнього усвідомлення (rig stong
zang thal), як пояснюється Дріме Озером у "Скарбниці дорогого Дхармадхату (chos
dbyings rinроche'i mdzod)". Іди ж стопами цього досконалого Будди і я буду радий!
\end{siderules}

\subsubsection{Об'єднання божества з реальністю}

Ви не повинні заспокоюватись на простому змішуванні свого розуму з божеством. Щоб
пов'язати божество з його істинною природою (don), необхідно зрозуміти, що це твій
власний розум, разом із зборами восьми свідомостей, виникає як форма та мудрість
божества. У цьому властивому стані самоусвідомлення (rig ра) є пробудженим
розумом (бодхічиттою), що результує формою божества. Будучи різнорідними, вони
дозрівають як один смак, через нероздільність Стадій Зародження та Завершення – велике
Тіло Мудрості.

\subsubsection{Чотири цвяхи, що \\скріплюють життєву силу}
Але, якщо цей момент незрозумілий, того Стадія Зародження не відрізнятиметься від
форми накопичення звичних тенденцій, які закріплять звичайний стан
існування. Тоді ти відвернешся від шляху Визволення і твоє досягнення зведеться не
більш ніж до переродження злим духом або демоном, подібним до Рудрі! Ось чому вчать, що
"чотири цвяхи, що скріплюють життєву силу (srog sdom gzer bzhi)" мають величезну
важливість. У тридцять дев'ятому розділі "Тантри досконалої таємниці (gsang rdzogs)" сказано:

\begin{verse}
Мудрість чи мирське, якщо чотири цвяхи,\\
Ті, що скріплюють життєву силу, не вбиті,\\
Практика ніколи не буде Плідною.\\
Спосіб володіти життєвою силою всіх Славних,\\
Складається в розумінні одного моменту та здобутті життєвої сили всього.
\end{verse}

\begin{verse}
Також як Рахула ковтає сонце на небі,\\
Не ганяючись за відображеннями у тисячах водойм,\\
Для набуття життєвої сили їх усіх\
Чотири цвяхи, що скріплюють життєву силу, дуже важливі.
\end{verse}

"Цвях самадхи (tin nge 'dzin)" дозволяє перетворити сильну прихильність по
по відношенню до свого звичайного тіла у форму божества. Це відбувається завдяки
односпрямованому зосередженню (sems rtseg cig) на формі божества, як на опорі,
використовується для візуалізації, а також завдяки подальшому його розвитку. У тантрі
сказано:

\begin{verse}
Цвях самадхи, це односпрямоване зосередження розуму.
На Тілі божества та на капалі без відволікання.\\
Опануй трьома б'єктами і зустрінься\\
З Великими Славним віч-на-віч.\\
Інакше херуку не побачиш.
Без цвяха постійного самадхи
Ніколи не побачити Просвітленого Тіла.
\end{verse}

Трьома об'єктами опановують таким чином. На початку божество ясно
є як об'єкт думки (bsam pa'i уul), потім ясно присутній як видимий об'єкт
(mthong ba'i yul) і, нарешті, воно ясно присутнє як відчутний об'єкт (reg pa'i yul).
Божество є розумом, що дозрів як чудова форма (sku), зустріч з якою називається
"досконалою візуалізацією". У "Радах Велико славного (dpal chen zhal lung)" сказано:

\begin{verse}
Усі форми божеств мудрості\\
На початку прояснюють як об'єкти розуму.
Потім розвивають силу їхнього актуального прояву,\\
А потім, завдяки силі абсолютно чистого розуму,\\
Їхня природа проявляється як чуттєвий об'єкт.
\end{verse}

\begin{verse}
І, нарешті, дві істини стають нероздільними;
Досягається досконала гнучкість тіла та розуму,\\
А божество ясно проявляється, як відчутний об'єкт.
Так ти долаєш явний вияв нечистого тіла.
\end{verse}

Що стосується "цвяха сутнісної мантри (snying ро sngags kyi gzer)", то слідує
односпрямовано зосереджуватися, коли корінна мантра чи обертається навколо
життєвої сили в Серці (thugs srog) або обертається туди і назад. Потім, коли
підраховується мантра, видих, вдих та затримка дихання очищаються як сутність
просвітленої мови. У тій самій тантрі сказано:

\begin{verse}
Як цвях сутнісної мантри повторюй мантру\\
Головного божества всіх зборів,\\
Як кореневу мантру навколишню життєву силу у Серці ХУМ.\\
Візуалізуй випромінювання та поглинання, а також повторюй мантру.\\
\end{verse}

\begin{verse}
У цей час з'являються слава, процвітання та пророцтва.
Потенційна сила стане досконалою, а божества зберуться.
Без досконалості Наближення через цвях сутності життєвої сили\\
Божества та пов'язані обітницею Дамчени не з'являться.
\end{verse}

Третє - "цвях незмінного розуміння (Стану?) (Dgongs ра mi 'gyur ba'i gzer)".
Неважливо як ви практикуєте, з Великим Славним [Херукою] або з будь-якими іншими
мирними та гнівними божествами, ви повинні зрозуміти, що божество є не що інше, як
ваш власний розум. Разом з цим розумінням божество дозріває у свою сутність,
стаючи одним смаком із Дхарматою (Реальністю) (chos nyid). І ці два стають великим
рівністю. У тантрі сказано:

\begin{verse}
Як цвях незмінного розуміння Великославний [Херука],\\
А також усі збори Мирних та Гнівних божеств\\
Є нічим іншим як своїм розумом.
Навіть Рудра є нічим іншим як розумом.
\end{verse}

\newpage

\begin{verse}
Будучи порожнім, сам розум є Дхармакая.
Будучи ясним і виразним, він Самбхогакая.
І, виявляється природним чином, багатьма способами, він - Нірманакая.
Прояви — це чоловічі форми, а порожнеча — жіночі.
\end{verse}

\begin{verse}
Нероздільність прояву і порожнечі - це Тіло Мудрості.
Неймовірні думки, а також спогади\\
Є повністю досконалими, почетом Славного.\\
Вони не описуються та просвітлені спочатку,\\
\end{verse}

\begin{verse}
Оскільки вони ніколи не відхиляються від цієї природи.
Без цвяха постійного розуміння,\\
Досягнення ('grub) Великого Славного\\
Не призведе до найвищого Досягнення,\\
Лише до звичайного, подібного до Рудрі.
\end{verse}

Четверте - "цвях випромінювання (випускання) і поглинання (phro 'du'i gzer)". Ця
велика чудова ілюзія (phrul chen ро) передбачає освоєння кожної з чотирьох видів
просвітленої активності (phrin las). Це також необхідно, як біла бура для алхімії
(Bzhu brtul). У тантрі дається таке пояснення:

\begin{verse}
Як цвях випромінювання та поглинання\\
Візуалізація використовується для трансформації.
Знаючи це, всі бажання здійснюються.
Завдяки самадхи випромінювання та поглинання.
\end{verse}

\begin{verse}
Потенціал розуму наводиться до досконалості,\\
І все, що видається, — з'являється.
Силою благословення мантр та мудрий,\\
Усі активності, що допомагають та шкідливі, будуть здійснені,
\end{verse}

\begin{verse}
Умиротворюючі, збільшують, \\
контролюючі та гнівні,\\
Подібно до того, як звеличення \\
та підношення однієї коштовності\\
Приводять до обставин \\
здійснення всіх бажань.\\
Якщо ж цвяхи різних
випромінювань та поглинань немає,
\end{verse}

\begin{verse}
Тоді [ні?]་один об'єкт зосередження\\
не приведе до мети.
Але якщо ти знаєш, як переходити
від поширення до збирання,\\
Тоді навіть нижчий демон тедранг
Виконуватиме безперешкодну активність \\
та здійснювати всі дії,\\
Оскільки все є чудовим творінням розуму!
\end{verse}

Коли основні моменти цих настанов про чотири цвяхи засвоєно,
Просвітлене Тіло буде безпосередньо сприйняте, і ти набудеш його життєву
силу. Знайди справжню мову, і ти набуваєш життєвої сили просвітленої мови. І коли
буде контролюватись сама Реальність, ти набудеш життєву силу просвітленого
Розуму. Твої три брами з'єднаються з просвітленим Тілом, Мовою і Розумом мирних і гнівних
сугат, і різноманіття стане одним смаком (rо gcig). Зроби це, і ти здобудеш
тисячоразову життєву силу Сансари та Нірвани. Так ти очистиш уроджений
потенціал (rtsal) (енергію?) Мудрості та освоїш різні види просвітленої
Активності.

\begin{siderules}
П'ята тема зрозуміла ясно. Шоста тема стосується зв'язку божества з реальністю, яка
також називається "розвиток (bogs 'don) Стадії Зародження" та "зв'язок через
активності найближчої причини (nye rgyu spyod pas mtshams sbyar)". Практика сутнісних
момент у медитації Стадії Зародження, як було описано вище, передбачає входження
у йогічні практики з особливими тимчасовими періодами та кількістю. Це дозволяє
вийти за межі мирських шляхів та отримати безпосередній зв'язок з шляхом бачення,
вищим духовним здійсненням Великого друку (phyag rgya chen ро mchog gi dngos grub).
Саме це мається на увазі під терміном "вихід за межі у вигляді поведінки".
У "Блазі колісниці (shing rta bzang ро)" сказано, що, незважаючи на вищий шлях Стадії
Зародження та інші [практики] самі по собі вони не призводять до звільнення, оскільки
необхідні інші чинники, що розвивають практику. Таким чином, хоч і мають
вісьмома заходами ясності та стійкості на шляху Стадії Зародження, але якщо не використовують
сутнісні настанови (man ngag) чотирьох цвяхів, які скріплюють життєву силу
для розвитку її за допомогою поведінки, тоді буде неможливо досягти результату цього
процесу, стану відьядхари.
\\
Тому в контексті зв'язку божества з реальністю ви повинні медитувати на шляху
певної досконалості (lam nges rdzogs). Отже, потрібно освоювати різні
стадії, від самадхи великої порожнечі до складної форми повного кола мандали, а також
необхідно досягти досконалості восьми заходів ясності та стійкості. Це називається
"шляхом з певної поведінки з періодами та кількістю (spyod pa'i dus grangs nges pa'i lam).
Як сказано:

\begin{verse}
Досягають могутності та вищого стану\\
Протягом шести, чотирнадцяти чи шістнадцяти місяців.\\
\end{verse}

Таким чином, за допомогою перетворюючих ритуалів (spog cho ga) просто неможливо не
досягти рівня відьядхари протягом шести місяців.\\
\\
І навіть якщо таке неможливо, сказано, що ті, хто присвячують себе практиці, оскільки їх
шлях також потребує ясної мети для навчань, до яких вони залучені. Такий тип
практикуючих на початку розвиває правильне розуміння ключових моментів щодо
Стадії Зародження та завершення, а потім, з самого початку вони пов'язують свою практику з
підходом та здійсненням, спираючись на сутнісні настанови чотирьох цвяхів,
які скріплюють життєву силу. Таким чином, у нашому бутті, відповідно
розумовим нахилам індивідуума, народжуються різні якості.\\
\\
Все інше рівносильне відходу від основних моментів Зародження та Завершення. Просто
називати свою практику "наближенням і досягненням" і залишатися в ретриті на
протягом років не призведе ні до чого, крім труднощів. Начитування сотень мільйонів
мантр не викликає навіть медитативного тепла (drod) звичайних якостей, які є
знаками розвитку на дорозі. Іншими словами, якщо сутнісні моменти Шляху не
беруться до уваги, що старанність не приведе ні до чого, крім погоні за міражем.\\
\\
У цьому контексті цвяха зосередження (самадхи) на божестві згадується " освоєння
трьох об'єктів (yul gsum gyad du gyur pa)". І в тексті сказано: Спочатку божество ясно
присутній як об'єкт думки (bsam ра'i yul), потім ясно присутній як видимий
об'єкт (mthong bа'i yul) і, нарешті, воно ясно присутнє як відчутний
об'єкт (reg ра'i yul). І це можна сформулювати інакше: "Спочатку ясно присутній
як об'єкт розуму (уid), потім ясно присутній як об'єкт органів чуття (dbang ро) і,
нарешті, ясно є як об'єкт тіла (lus)".\\
\\
На початку слід досягти знайомства (goms ра) із проявом божества. На цьому етапі
є божеством лише в концептуальному чи вербальному значенні (bsam ра'і blo rtog gam ngag kyi brjod pa).
Іншими словами, божество ясно тільки з точки зору аналітичного розуму,
і при цьому необхідно скористатися думками певним чином (rtog dpyod kyi blo ngor).
Це називається "ясно бути присутнім як об'єкт думки" або "ясно бути присутнім як об'єкт розуму".
Коли знайомляться з проявом божества і досягають певної стійкості у цьому
В процесі, більше немає необхідності запечатувати медитацію (rgyas 'debs) думками,
використовуваними на попередній стадії. Натомість божество стає якби
видимим для повністю функціонуючих чуттєвих здібностей ока. Не втрачаючи з
виду еліксир безтурботності без думок (rtog med zhi gnas kyi rtsig), прояви
візуалізації стають виразними, аж до зіниць очей божества. Саме це
мається на увазі під "ясним присутністю як видимого об'єкта" або "ясні присутністю
як об'єкта органів чуття".\\
\\
Наприкінці цього процесу Стадія Зародження повністю освоюється. І коли досягають
стану зрілого відьядхари, більше не сприймають звичайні нечисті прояви,
оскільки вони зливаються з мандалою божества. У цей момент актуалізується ілюзорне
тіло (sgyu lus), об'єднана форма божества (zung jug gi lha sku). Саме це
мається на увазі під "ясним присутністю як відчутного об'єкта" або "ясним присутністю як б'єкта тіла".
Настанови про чотири цвяхи, які скріплюють життєву силу, є ключовими.
моментами, які дозволяють опанувати тисячочасткову життєву силу Сансари і
Нірвани. Ці сутнісні настанови нерозривно пов'язують звичайні тіло, мову, розум і
активності, що сприймаються в Сансарі, з просвітленими Тілом, Мовою, Умомі
Активностями, пов'язаними зі Станом Будди, скріплюючи їхню життєву силу подібно
цвяхів. Коли освоюють ці моменти, то досягають повної влади над Сансарою та
Нірвана, а також нечисті тіло, мова і розум очищаються і трансформуються в
просвітлені Тіло, Мова та Розум. З цього моменту здійснюють Діяння Будд (sangs
rgyas kyi mdzad pa) за допомогою чотирьох типів неймовірної просвітленої активності
(phrin las) - умиротворюючою, збільшує, підпорядковує і гнівною (zhi rgyas dbang
drag). Таким чином, чотири цвяхи, що скріплюють життєву силу, містять ключові
моменти шляху Стадії Зародження.

\\
Сказано, що шляхи, позбавлені цих принципів, подібні до "бігу на місці", тобто така
активність приносить лише втому. Правда полягає в тому, що ці шляхи не призводять ні до
чому, крім втрати сил. Крім того, хоча ми класифікуємо божеств, на яких
медитуємо, як божеств мудрості або мирських, цей поділ насправді робиться на
На основі того, була або не була осягнута (rtogs) порожнеча. Іншими словами, поки
мають цвях незмінного розуміння Реальності (Дхармати), навіть візуалізація себе в
формі мирського демона дасть силу досягти найвищого духовного досягнення (dngos grub).
З іншого боку, ті, хто не мають цвяха незмінного розуміння, впевнені в тому, що
вони і божество різні. Такі індивідууми не отримають найвищого духовного досягнення,
незважаючи на медитацію на позамежному божестві йідамі. Адже вони зайняті тим, що
вправляють свій розум бути гнівним та пристрасним. Незабаром їхній розум стає впертим, як
у лютих варварів, які захоплені ритуалами жертвоприношень та іншими
безсердечними практиками. Ці люди насправді пов'язані у своїй практиці з могутніми
мирськими духами, під чий вплив вони підпадають. І хоча вони можуть називати це Таємною
Мантрою, насправді це називається Вченням Демонів, від якого застерігає
"Калачакра-тантра". У такому підході виразно збиваються зі шляху і закінчують тим, що
стають демоном чи духом.
\end{siderules}

\subsection{Привнесення переживань на шлях}

Сьома тема містить дві частини: десять речей (shes par bya ba), які слідують
зрозуміти і шість основоположних обітниць самої (dam tshig). Щодо першої сказано,
що Таємна Мантра повинна практикуватися подібно до десяти пізнаваних речей, починаючи з
ілюзії міражу. Таким чином, говориться, що Таємна Мантра має практикуватися з
десятичастним розумінням. І це стосується всіх випадків споглядання мандали божеств,
у медитативному чи постмедитативному станах.\\
\\
Десятичастне розуміння таке:

\begin{tabular}{ll} 
1 & Садхани подібні ілюзіям;\\ 
2 & всі назви і слова не мають сутності (snying ро),\\ 
& подібно до міражу, який обманює дикихжи вотних;\\ 
3 & всі активності подібні до снам;\\ 
4 & всі речі позбавлені істинної природи, \\ 
& подібно до відображень у дзеркалі;\\ 
5 & ​​всі місця і землі подібні до міст гандхарвів;\\ 
6 & всі звуки порожні від самобуття (ngo bo), подібно до еху;\\ 
7 & всі чудові форми подібні до відображень \\ 
& Місяця у воді - проявляються, \\ 
& але немає справжнього існування;\\ 
8 & різні медитативні занурення самадхи \\ 
& подібні до бульбашок на воді;\\ 
9 & всі випромінювання та поглинання різноманітні, \\ 
& подібно до оптичних ілюзій;\\
10 & всі чудові прояви виникають у різний спосіб,\\ 
& але не мають власних характеристик (mtshan nyid), \\ 
& подібно до магічних ілюзій.\\
\end{tabular}

Друга тема пов'язана з шістьма основними обітницями та самаями:\\

\begin{tabular}{ll}
1 & ніколи не переставати прагнути до вчителя, \\ 
& який дає сутнісні настанови (man ngag); \\
2 & намагатися створити сприятливі умови для \\ 
& медитації та відкидати несприятливі;\\
3 & не дозволяти згасати своєму медитативному зануренню, \\ 
& чим би не займався; \\
4 & не відмовляйся від свого божества; \\
5 і 6 & зберігати суть своєї медитації та \\ 
& поведінки в таємниці від тих, \\ 
& хто не є відповідним учнем.
\end{tabular}
\\
\subsection{Результати шляху, пов'язані з [Рівнем] \\ Чотирьох Відьядхар - Тримачів Ведення}

\subsubsection{Наближення}

У другому розділі пояснюються рівні досягнення, пов'язані з [рівнем] Чотирьох
Відьядхар, у відповідність з особливим шляхом поділу цих чотирьох рівнів на чотири частини
Наближення та Досягнення (bsnyen sgrub). Все це стосується практики Наближення, в
якої розум зосереджують на тонкому самадхи, єдиного друку (phyag rgya gcig) та інших
аспекти, що залучаються для візуалізації трьох об'єктів. Така медитація є
причиною того, що виявляються якості, пов'язані з Шляхами Накопичення та З'єднання.
З взаємозалежного ланцюга, пов'язаного із застосуванням уважності (dran ра) та іншого
тонкі енергії п'яти елементів входять до центрального каналу як переживання цього знання
(shes bya'i nyams su). Є певні знаки, які зазначають виникнення цього
процесу, включаючи прояви диму, міражів, спалахів та безхмарного неба (smig rgyu dang
srin bu me khyer dang sprin med kyi nam mkha'). Все це слід розуміти як знаки того, що
починають розгортатися прояви медитативних переживань.

\\
Крім того, є деякі знаки, які з'являються уві сні. Неодноразово бачити
себе оголеним це знак того, що звичні тенденції очищаються. Сходження по
сходи в небо - це знак здобуття досягнень (thob ра). Сидіти верхи на сніговому леві
чи слоні — це знак подолання перешкод (non ра). Є також знаки, пов'язані з
отриманням передбачень, наприклад, образи божеств, що з'являються, можуть посміхатися. Якщо
коротко, то у "Ваджрній структурі (rdo rje bkod ра)" пояснюється, що є неймовірне
кількість знаків, які вказують на розвиток по дорозі. Однак, радіти їм як
знакам своєї переваги, означає піддаватися впливу демонів. Тому слід робити
все, що б залишатися вільним від відшкодувань на щось позитивне.\\
\\
І так само, коли медитують на стадії зовнішньої спеки (phyi'i drod), то бачать
дрібні частинки, літери, символи, атрибути рук, тонкі форми (rdul phran dang yi ge
dang phyag mtshan dang sku phra mo) та інші знаки. Сприймаються також кольори та об'єкти
(kha dog dang skye mched), пов'язані з вогнем, водою, вітром та іншими подібними
факторами. На стадії внутрішньої спеки, внаслідок медитації на божество, рух
дихання (dbugs) стає невідчутним. На стадії таємного жару усі об'єкти (spyod yul)
осягаються як ілюзорні. Без необхідності узгоджених зусиль без опори на
причини та умови, всі явища (chos thams cad) осягаються як просвітлені всередині
базового простору (простору основи?) (dbyings). В результаті цього все
переживання ясно виявляються як Мудрість.
\\
Є ножі знаків, які виникають у цьому процесі медитації. Зовнішніми
знаками є прояви світла, звуків і запахів. Форми божеств можуть сміятися,
масляні лампи можуть самі собою спалахувати, а капали можуть злітати в повітря. Крім того,
можна насолоджуватися хорошим самопочуттям та відчуттям захоплення. Внутрішні знаки
полягають у збільшенні співчуття, ослабленні прихильності, неупередженому відношенні,
більшої уважності по відношенню до обітниць самої, а також у доброму становищі по
по відношенню до Гуру і до друзів по Дхармі. Також можуть слабшати надії на позитивний
результат, прихильність до Сансарі та страх демонів. Як і раніше слід відкинути
відчуття радості від того, що отримуєш щось подібне.
\\
Внаслідок досягнення вершини (rtse mo) з'являються різні знаки, які
відзначають змішання розуму та проявів. Наприклад, п'ять затьмарень-кльош більше не
виникають у зв'язку із зовнішніми об'єктами, а також п'ять зовнішніх елементів більше не можуть
пошкодити тілу.
\\
На стадії терпіння (bzod ра) всі прояви стають податливими (м'якими), а
також виникають деякі знаки того, що з'являється контроль над розумом та проявами.
Наприклад, з піску тоді може з'являтися золото, а також вода із сухої землі та дерева з
деревного вугілля. При цьому тіло не обов'язково змінює свій звичайний стан, але розум
дозріває (smin ра) у форму божества.\\
\\
Ця стадія називається «Повністю дозрілим Відьядхарою» (від'ядхарою дозрівання)
(rnam par smin pa'i rigs 'dzin)". У "Обширній ілюзії (sgyu' phrul rgyas pa)"
сказано:

\begin{verse}
Подібно до сургучу і рельєфу на пресі,\\
Велика Друк, перед своїм досягненням,\\
Є чимось іншим як досконалою і могутньою формою.
\end{verse}

У цьому уривку "рельєф на пресі (rgya mig)" є метафорою тіла (lus), тоді як
"Сургуч" відноситься до розуму. Сенс цього пов'язаний із здійсненням Великої Друки
(махамудри). І в цьому уривку вказується, що якщо вмирають на цій стадії, недосягнувши
вищого стану (chos mchog), то Махамудра досягається в проміжному стані-бардо,
подібно до того як глиняна фігурка (sa' tsa tsha) виходить з форми. Причина цього
полягає в тому, що тіло, яке народжується (народилося? пр.вр.?) в результаті дозрівання
карми, буде відкинуто, а розум перетвориться на форму божества. В "Етапах шляху" сказано:

\begin{verse}
Коли Наближення практикують упродовж шести місяців, \\
Ваджрне тіло поки що не досягається.\\
Внаслідок невеликих зусиль, а також деяких умов та доброзичливостей,\\
Залишкове тіло, обумовлене концепціями, залишається.\\
Завдяки усвідомленню слідують (йдуть?) до стану Ваджрадхари.
\end{verse}

З іншого боку, коли досягається потужний стан самадхи, пов'язаний із рівнем
вищого стану, отримують повний контроль над життєвими процесами. У тантрі сказано:

\begin{verse}
Коли досягається стан відьядхари,\\
Наділене силою довгого життя,\\
То знаходять контроль і досягають вищого стану.
Протягом шести, дванадцяти чи шістнадцяти місяців.
\end{verse}

\subsubsection{Близьке Наближення}

Йдеться про те, що шлях бачення (mthong ba'i lam) здійснюється під час практики.
Близького наближення (nye bar bsnyen ра). У тантрі сказано:

\begin{verse}
Досконалість підношення, що включає все оточення,\\
Докладай зусиль, як у традиції (стадії) Наближення.
\end{verse}

На цій стадії, незалежно від того, як медитують, від єдиного стану
великого таїнства (gsang chen rigs gcig) і до сконструйованого друку (phyag rgya spros
ра), ясність та стійкість візуалізації поєднується з Реальністю, природою великого
блаженства (chos nyid bde ba chen po'i don). Така практика призводить до того, що насправді
У справі проявляються якості семи факторів просвітлення (byang chub yan lag bdun gyi yon tan).
Коли розум досягає шляху Бачення, виникають п'ять форм ясновидіння (mngon par shes ра),
чотири чудові еманації (cho'phrul) та інші здібності, пов'язані з цією стадією.
Крім того, з'являється здатність чути Вчення-Дхарму безпосередньо від Будд.
Нірманакаї. Далі, коли стає очевидною справжня природа об'єктів (don rang gi mtshan
nyid mngon du byas pa), великий та вищий стан доводиться до своєї досконалості
(mthar phyin), а забруднене тіло перетворюється на Ваджрне Тіло. І оскільки ця форма
(стадія?) вільна від народження та смерті, вона називається "Відьядхарою контролюючим
тривалість життя" (tshe la dbang ba'i rigs 'dzin). В "Етапах шляху" сказано:

\begin{verse}
Разом із здобуттям досягнення восьми зборів,\\
Природу бачать, до неї входять і досягають досконалості.
Тому забруднення тіл, світів
І місць народження добігають кінця.\\
Стаючи Ваджрним Тілом, \\
сім'єю життя (tshe yi rigs),\\
Все, що бачать, є Нірвана.
Без відкидання тіла досягається \\
рівень Стану Будди.\\
Усі страхи зникають, а чудові еманації \\
стають досконалими.
\end{verse}

Дана стадія здійснення (sgrub) подібна до тієї, що великий ачарья
Падмасамбхава досяг у печері Маратика, де Мандарава була його партнеркою з практики,
і де він досяг стану, вільного відродження та смерті, за допомогою практики
безпосередньої причини (nye rgyu'i spyod ра).

\subsection{Досягнення}

Практика Досягнення (sgrub ра) описується як шлях медитації, що вдосконалює.
(Sgom pa'i lam). Щодо цієї теми в тантрі сказано:

\begin{verse}
За допомогою досягнення (здійснення) старайся на шляху медитації,\\
І тоді досягнеш божественного Райдужного Тіла Махамудри.
\end{verse}

Щодо характеристик, то природа цього медитативного стану вільна від
помилок ('khrul ba) і тому подібна до стану Будди. Однак між ними є
різниця у зв'язку із постмедитативним станом (rje sthob). На цій стадії енергія,
пов'язана з рівністю Реальності (chos nyid mnyam par rtsal) все ще потребує
очищення. На основі мандали тіла різні чудові еманації досягаються в стані
нерухомого медитативного зосередження (mi g.yo ba'i ting nge 'dzin). Внаслідок цього
природа мандали тіла проявляється у райдужній формі, а ментальні забруднення, пов'язані
з дев'ятьма рівнями-бхумі (sa dgu), очищаються. Далі, восьмичастковий шляхетний шлях
повністю удосконалюється у формі Ясного Світла Мудрості (ye shes 'od gsal), вільного
від тонких характеристик (mtshan mа). Завдяки цьому можуть отримувати як ментальні, так і
символічні вчення від Самбхогакаї. У тантрі сказано:

\begin{verse}
Наше тіло стає печаткою переможних,\\
І божество стає явним за допомогою медитації.
Прикрашене головними та вторинними Ознаками,\\
Воно є Відьядхарою Махамудри - Великої Печатки.
\end{verse}

У цій категорії є різні підрозділи. Перебувають на рівнях з другого по
п'ятий відомі як "Відьядхар ваджри". Оскільки їх розуміння є ваджрним,
воно знищує забруднення кожного окремого рівня. Перебувають на шостому рівні
зосереджуються на практиці Досконалості мудрості (shes rab kyi pha rol tu phyin pa) та
повертають колесо Дхарми. Саме тому вони називаються "Відьядхарами Колеса
[Вчення]". Ті, хто перебуває на сьомому рівні, називаються також, оскільки завдяки своїм
майстерним засобам вони вміло працюють подібно до колеса. Перебувають на восьмому рівні
мають неконцептуальну мудрість (mi rtog pa'i ye shes), подібну коштовність, і
досягають контролю над своїм абсолютно чистим потенціалом (rang gi khams). Тому
вони називаються "Відьядхарами коштовності". Перебувають на дев'ятому рівні
називаються "Відьядхарами лотоса", оскільки вони вільні від прихильності і можуть
створювати чисті світи (zhing sbyong), а також працювати на благо інших. Перебувають на
десятому рівні роблять досконалою свою просвітлену активність, завдяки якій
вони приносять благо всім живим істотам. Саме тому їх називають "Відьядхарами
меча". В "Етапах шляху" сказано:

\begin{verse}
Друга Самбхогака, сім'я друку,\\
Складається з відьядхар\
Ваджри, колеса, коштовності, лотоса та меча.
\end{verse}

Прикладом цього рівня досягнення є великий ачарья Падмасамбхава, коли
він досяг реалізації в Янглешо, що в Непалі. Там він, спираючись на мандалу Славного
Вішуддха Херукі продемонстрував те, як досягти рівня Відьядхари Махамудри.
Великої Печатки. І це також те місце, де він досяг четвертого типу просвітленої
активності.

\subsection{Велике досягнення}

Четверте - це практика Великого Досягнення (sgrub ра chen ро), яка
описується як Стадія не-навчання (mi slob pa'i lam). У зв'язку з цим сказано:

\begin{verse}
Коли знаки стають стійкими,\\
Демонструй Велике Досягнення.
\end{verse}

На цьому етапі попередні шляхи були повністю пройдені, і акцент робиться на
групової практиці, яка дозволяє зборам, разом із сотнею тисяч, об'єднатися з
просвітленим станом. У цій мандаліті, хто ставиться до спонтанної родини (lhun gyis
grub pa'i rigs), здебільшого рівні Буддам щодо вищих якостей, якими вони
мають. Тим не менш, у традиції Мантри є миттєвий розвиток, який
відбувається в ході індивідуальних фаз (bye brag phyed ра) цієї стадії, внаслідок чого
процес тренування доводиться до досконалості (mthar thug). І коли це відбувається, то
досягається Стан Будди без тренування, а також безпосередньо зустрічаються з
Дхармакаєй. Іншими словами, на вершині цього стану досягають відкидання та
розуміння (spangs rtog). Саме так переможний Падмасамбхава описує [рівень]
"Відьядхари Спонтанної Присутності (lhun grub rigs 'dzin)". В "Етапах шляху" сказано:

\begin{verse}
У результаті вдосконалення сили попередніх сімей,\\
Як уже було пояснено, нечисте очищається,\\
А потрійне знання Стану Будди розгортається.
Так досягається спонтанно присутня сім'я.
\end{verse}

Може виникнути думка, що вважати спонтанну сім'ю, що належить
виключно до рівня Стану Будди не цілком правомірно. Однак це не так.
Основний намір використовувати чотири сім'ї відьядхар, який виявляється в
традиції мантр, полягає в уявленні п'яти шляхів традиції сутр. І це підтверджується
загальним змістом того, що внутрішній зміст тантр пояснюється за допомогою шести меж
та чотирьох принципів (mtha' drug dang tshul bzhi). Таким чином, Велике Досягнення, як
це пояснюється в контексті просування через чотири розділи Наближення та Досягнення,
також вважається метою практики, в якій нема чого усувати або досягати. Факт полягає в
тому, що цей розумний сенс усуває будь-яке подібне протиріччя.
\\
Цей спонтанний відьядхар класифікується як рівень стану Будди. Тим не
менш є подальші рівні, які досягаються, коли природа цієї стадії досягає
повної сили. Вищий рівень, подібний до лотосу неприв'язаності, виникає, коли
перебувають не забрудненими жодними негативними вадами, незалежно від того як
безвідносне знання використовується під час аналізу. Це також стосується рівня Великого
накопичення колеса - чакри складів (yi ge 'khor lo tshogs) та різних невимовних
просвітлених якостей, таких як тридцять два піднесених головних ознак і
вісімдесят другорядних. І, нарешті, також є тринадцятий рівень Ваджрадхари, на
якому радіють невичерпному колесу прикрас (mi zad pa'i 'khor lo), просвітленим
Тілу, Речі та Уму Татхагати. В "Рашних зборах прикрас (rgyan stug ро bkod ра)"
сказано:

\begin{verse}
Незважаючи на те, що вони пробуджені.
У вищій сфері Аканіштхі,\\
Досконалі Будди не здійснюють\\
Свою просвітлену діяльність у сфері бажання.
\end{verse}

Як говорилося раніше, індивідуальні рівні традиції мантри перетинаються в
одному моменті переживання (spyod yul skad cig ma), завдяки якому Просвітлення
втілюється в Аканіштху та здійснюються незлічені просвітлені активності,
такі як дванадцять дій безперешкодних вправних засобів (thabs mа' gags pa'i
mdzad ра). Ці класифікації робилися у світлі причинних зв'язків, що створювалися
автоматично (rang gir byed ра) і пояснювалися як є насправді. У тантрі сказано:

\begin{verse}
Коли досягають вершини, спонтанного здійснення,\\
то діють як регент-\\
Справді повертаючи таємне колесо,\\
Уча відповідної Дхармі всіх,\\
А також демонструючи дванадцять активностей (Дій).
\end{verse}

Прикладом цього рівня досягнення є великий ачарья Падмасамбхава у його
теперішньому стані, в якому він перебуватиме до закінчення Сансари. Перебуваючи на
На рівні Відьядхари спонтанної присутності, Падмасамбхава не відокремлений від усіх Будд. Він
вчить Дхармі своє чисте оточення у сфері Аканіштха Лотосового світла і ніколи не
втомлюється проявляти співчутливу активність заради блага живих істот.
\\
Водночас сам великий майстер наділяє відьядхару силою контролювати
тривалість життя, а також Великою печаткою. Таким чином, це не описувалося
більшістю з його послідовників. Однак те, що я пояснив тут, відповідає моєму
власного осягнення, яким я завдячую другому Будді (Падмасамбхаве), чиї промені
співчуття змусили розпуститися лотос мого розуму. \\

\\
Деякі вважають, що Махамудра-Велика Друк охоплює з першого по
сьомий рівні-бхумі (sa), тоді як стан спонтанної присутності (lhun grub)
відноситься до трьох чистих рівнів. Однак, це не так. І це підтверджується щойно
процитованою цитатою з "Етапів шляху". У зв'язку з цим Всеведучий писав:
"Твердження деяких вчених, що стан Махамудри Великої Печатки охоплює з
першого по сьомий рівні, і що стан спонтанної присутності відноситься до трьох
чистим рівням, відповідає неправильному розумінню. Причина цього полягає в тому,
що розвиток чотирьох видів Відьядхар охоплює всю стадію початківця аж до
Стану Будди". \\
\\
Індивідууми, які в основному вже зібрали два накопичення та досягли великої
сили щодо їх знання та медитативного занурення можуть проходити через ці шляхи
більш безпосередньо. Такі індивідууми миттєво переходять із Шляху З'єднання на
Шляхи Бачення та Медитації. Завдяки цьому, вони слідують довершеного шляху (mthar phyin
pa'i lam). Також є деякі з винятковими здібностями, які переходять
прямо з Шляху З'єднання до Стану Будди. У тантрі сказано:

\begin{verse}
Деякі вдосконалюють п'ять істинних Тіл у шістнадцять.
Від стану освоєння (dbang sgyur rigs),\\
Тоді як інші розвиваються від стану Махамудри - Великої Печатки
До неперевершеного стану Самантабхадри.
\end{verse}

Зважаючи на це, може виникнути питання, чи не досягав другий Будда-Ачарья
Падмасамбхава, реалізації поступово, як зазначається у прикладах. Однак, це не так,
оскільки в таких текстах як "Просвітлення Вайрочани (rnam snang mngon byang)" та
"Таїнства краплі місяця (zla gsang thig lе)" вказується:

\begin{verse}
У блаженній сфері, відомій як Аканіштха,\\
Будди стають повністю просвітленими,\\
Та тоді випромінюють Просвітлення тут.
\end{verse}

Так було і з Буддою Шакьямуні. Хоча він став просвітленим і досконалим,
досягнувши відкидання і осягнення незліченні кальпи тому, проте, він
продемонстрував Дванадцять Дій (mdzad ра bcu gnyis) у цьому світі.\\

\begin{siderules}
У другому розділі пояснюється, як слідують шляхам і рівням відьядхри, а також
обговорюються чотири розділи Наближення та Досягнення. Втім, ці чотири практики
можуть застосовуватись у різних контекстах. Вони можуть застосовуватися в контексті одного
ритуалу, або ж з часом та кількостями, пов'язаними з практикою. Будь-який варіант
прийнятний. Однак у даному контексті вони описуються у зв'язку зі стадіями та шляхами.
На шляху накопичення знайомляться та звертаються до божества Посвяти. Потім, не роблячи
нічого, що суперечить обітницям сама, які є життєвою силою Посвяти,
божество-йідам розглядається як невіддільне від нашого розуму. Коли чотири практики
Шляхи Накопичення завершено, посилюють практики, що відповідають Шляху З'єднання.
Приступивши до таких практик актуалізують знаки, що відзначають стадії жару,
вершини та прийняття (drod rtse bzod). Далі також актуалізується велика вища
стан (chos mchog chen ро), «Відьядхар повного дозрівання» (rnam smin rig 'dzin).
Таке наближення.
\\
На Шляхи Бачення справжнє Ясне світло виникає безпосередньо в нашому потоці розуму.
Потім вогонь мудрості очищає елементи та нечисте тіло. У цей момент ще більше
наближаються до божества йідаму, виявивши стан, що виходить за межі
народження та смерті. Таке Близьке Наближення.
\\
Далі, на Шляху Медитації актуалізують стан Відьядхари Махамудри, який за
формі тотожному божеству йідаму. Таким є досягнення, що означає, що наше
власне благо та благо інших здійснюються одночасно.\\
\\
На шляху звільнення зусилля, які докладають заради власного блага, припиняються.
У постмедитативному стані (rjes thob) на цій стадії просвітлені активності,
які здійснюють для добра інших, не відрізняються від активностей Татхагат. Таке
Велике Досягнення.
\\
У контексті шляху накопичення, тонкі енергії п'яти елементів входять до центрального каналу,
породжуючи п'ять звичайних знаків. Коли вітер землі входить до центрального каналу — з'являються
такі знаки, як дим (du ba); коли входить вітер води - з'являються міражі (smig sgyu); коли
входить вітер вогню - з'являються іскри (mе khyer); коли входить вітер вітру – з'являються
вогники (mar me); коли входить вітер простору – з'являється простір. Ці п'ять
є знаками, які виявляються безпосередньо органів почуттів. У цей момент
все, на що дивляться, проявляється як розпливчасте та туманне, подібно до диму; мерехтливе,
подібно до міражу; спалахує, подібно до іскор; помаранчеве (dmar ser), подібно
світильнику або подібно до безхмарного неба. Це відбувається лише в головному
медитативної рівноваги.

\\
Опис цих стадій спека, вершини та прийняття на шляху з'єднання легко зрозуміти. Однак,
в описі стадії вищого стану, що пов'язується з Відьядхарою повного
дозрівання, є важкі місця, які починаються зі слів: "Подібно до сургучу і рельєфу
печатки". Слів "рельєф" тут співвідноситься з вирізаним малюнком надруку, тоді як
"Сургуч" - це матеріал, який використовується для нанесення друку. Однак, у всіх
значення і значення, цей приклад схожий з прикладом вилучення глиняної фігурки з
форми. Сенс цього полягає в тому, що хоча розум може і дозріває у форму божества, проте
менше все ще немає виходу за межі фізичного тіла, яке є дозріванням
минулої карми.\\
\\
Далі слідує фраза: "Вищий стан контролю (dbang bsgyur rigs kyi dam pa)". І це
відноситься до того факту, що на цій стадії досягають Тіла світла (dwangs mа 'od kyi lus) як
опору, і навіть стають рівними доброю долею (skal ba) богам сфери форми. У цьому
контексті воно схоже на Всемогутнього царя (dbang phyugs), або на Владику цих чистих небес.
Так само можна описувати зовсім Просвітлене божество-йідама як "Всемогутнього".
У випадку «Відьядхари, який має контроль над тривалістю життя», причина
визначається як знання. Це стосується думки, що розглядає нероздільну
природу Сансари та Нірвани. І умова тут визначається як входження, тобто самадхи.
Отже, той, хто зробив досконалими чотири практики шляху накопичення, може
досягти шляхи бачення, використовуючи різні посилюючі практики, які
мають на увазі групу практик по дорозі з'єднання. \\
\\
У цей момент у цьому процесі виснажуються три забруднення (zag ра) і досягають
ненародженого та невмираючого Ваджрного тіла. Три забруднення - це забруднення тіла (lus),
сфери (khams) та місця народження (skye gnas). Перше передбачає неможливість
контролювати процес смерті та хвороби. Друге пов'язане з неможливістю використовувати
стану медитативного зосередження, щоб побажанню народжуватися у сфері бажання. І
третє передбачає неможливість контролювати процес народження одним із чотирьох
способів.
\\
Чотири дива (cho 'phrul) - це чудеса вербального виразу (kun brjod), відповідних навчань
(rjes bstan), магічних проявів (rdzu 'phrul) та маніфестації явищ (chos snang). І
хоча в "Стислому розумінні (dgongs 'dus)" представлені лише два різні види, цю
категоризацію дуже легко пояснити. Чудо вербального вираження пов'язане з
просвітленим розумом і в даному випадку співвідноситься свідомістю розумів інших істот,
завдяки власним психічним силам. Чудові прояви можуть змінювати
сприйняття інших істот, наприклад, вселяючи віру тим, хто її немає. Чудо придатних
навчань відноситься до просвітленої мови, у зв'язку з навчанням будь-якої з трьох Колісниць Дхарми,
яке найбільше підходить для підкорення учнів. І, нарешті, диво маніфестації явищ
передбачає здатність здійснювати такі речі як землетрус у шести
напрямах, а також маніфестація вісімнадцяти великих знамень (ltas chen ро) вчасно
вчення Дхарми.
\\
Шлях медитації пов'язаний із відьядхарою Великого друку. У цьому контексті з'являється
фраза: "завіси, пов'язані з дев'ятьма рівнями". Це стосується дев'яти звичайних рівнів
на шляху медитації, з другого рівня та вище.\\
\\
Спонтанно присутній відьядхар пов'язаний з шляхом Визволення, де ми знаходимо
фразу: "потрійна мудрість Стану Будди". Це є синонімом трьох типів
усвідомлення, яким навчають у "Сутрі великої гри (rgya che rol ра)", де сказано, що наш
Вчитель досяг трьох типів усвідомлення, коли він здійснив (mngon du mdzad ра)
Просвітлення. Перше — це знання минулих життів протягом неймовірного
проміжок часу. Другий тип - це знання смерті та народжень, тобто знання
обставин смерті та майбутніх народжень кожної істоти. Отже, мають
невимовним знанням смерті та народження. І, зрештою, є знання виснаження забруднень.
Це точне знання того, виснажуватися чи не забруднення, як щодо себе самого, так і
щодо інших.\\
\\
Шраваки, пратьєка Будди і бодхісаттви мають лише подобу цих трьох видів знання.
Тоді як у досконалого Будди ці три види знання досягають найвищого рівня і є
досконалими. Таким чином вони є безприкладними.\\
\\
Опис особливих знаків та прикмет (mtshan dpe) можна знайти у тантричних коментарях.
У сфері аканіштхи, що природно проявляється, різні сфери, чудові палаци,
центральні постаті, почти і т.д., виникають як чудова гра єдиної мудрості (ye shes
gcig gir nam rol). У цьому контексті шістнадцять істот другорядних класів та їх
чоловік і утворюють свиту. Такі знаки. Присутність п'яти божеств на короні (dbu rgyan)
кожного чоловічого божества представляє прикмети. Всі разом – це внутрішні знаки та
прикмети.
\end{siderules}
