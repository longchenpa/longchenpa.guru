\addtocontents{toc}{\protect\newpage}
\section{Чистий результат. \\ Досягнення стану єдності}

\subsection{Сутність Плоду: \\ Загальна вистава}

Практикуючи таким чином, самадхи порожнечі очищає чисту Мудрість, тоді як
візуалізація божества разом із самадхи подібним до музичного супроводу очищає
забруднення, які призводять до залучення до звичайних станів існування. Розуміючи,
що божество є природним проявом Мудрості, а не чимось
сприймається як зовнішній об'єкт, звичні тенденції, пов'язані з
концептуальним мисленням зникають. Далі, трансформація основи всього (kun gzhi gnas
'gyur) призводить до Мудрості Дхармадхату. Також і в "Сутрі входження до трьох Тіл (sku gsum
la 'jug pa'o mdo)" сказано:

\begin{verse}
Свідомість основа всього, \\
розчиняючись у Просторі, \\
утворює відбиваючу мудрість. \\
Ментальна свідомість, \\
розчиняючись у просторі,\\
утворює Мудрість рівності. \\
Забруднений розум, \\
розчиняючись у Просторі, \\
утворює Розрізняючу Мудрість.
І п'ять свідомостей органів чуття, \\
розчиняючись у Просторі, \\
утворюють Всеосяжну Мудрість.
\end{verse}

У цей момент відбувається безліч змін. На зовнішньому рівні прояви
трансформуються у Будда-сфери (zhing khams). Внутрішньо скандхи трансформуються в
форму божества, але в таємному рівні вісім зборів (свідомостей) трансформуються в
Мудрість. Це те, що розглядається як "Всемірно Просвітленим" (kun tu sangs
rgyas)". Як сказано в "Сімдесяти строфах притулку (skyabs 'gro bdun cu pa)":

\begin{verse}
Оскільки прокидаються від сну незнання\\
І розширюють розум, що охоплює все, що пізнається,\\
Стан Будди розпускається, подібно до пелюсток лотоса.
\end{verse}

Відповідно до підходу "Великої магічної мережі (sgyu 'phrul drwa ba chen po)",
П'ять Тіл одночасно досягають досконалості, як тільки актуалізується лотосовий
рівень неприхильності (ma chags padma сап), а також рівень великого накопичення
чакри складів (yi ge'khor lo chags chen). Тут я наводжу лише стислий огляд цих п'яти [Тел].\\

\subsubsection{Важкодра}

Незмінна Ваджракая описується в "Мережі Мудрості (ye shes drwa ba)":
Чистота простору - це Ваджракая,
Постійна, невимовна і немислима.
Єдиний шлях, який використовується всіма Буддами, це шлях Природного сяйва
(rang bzhin gyi 'od gsal), Початковий Простір (gdod ma'i dbyings). В Реальності (chos
nyid) кульмінація (mthar thug) цього процесу незмінна, а її ваджрна природа постійна
і не обумовлена, звідси і термін "ваджрака". Виходячи з розгляду того, що вона
природно чиста, спочатку вільна від забруднення, і навіть цілком чиста, у цьому
сенсі, що вона вільна від усіх форм випадкового забруднення (glo bur gyi dri та),
Ваджракая також називається "Станом Будди з подвійною чистотою (dag ра gnyis Idan gyi
sangs rgyas)".

\subsubsection{Абхісамбодхіка}

Що стосується Абхісамбодхікаї, то в тому ж тексті сказано:

\begin{verse}
Абхісамбодхіка описується\\
Як чиста і як вільна від нечистоти,\\
Досконала, оскільки якості розвинені,\\
А також вона єдина, бо нероздільна.
\end{verse}

Коли сяюча ясність природи розуму досягає своєї межі (sems nyid 'od du gsal
ba mthar thug), то вона стає частиною цієї подвійної чистоти. Що стосується
виявляється аспекту, то є всі унікальні просвітлені якості
Стан Будди. Сюди входять десять сил (stobs), чотири безстрашності (mi'jigs ра),
вісімнадцять окремих якостей Стану Будди (sangs rgyas kyi chos та 'dres pa), великий
співчуття (thugs rje chen ро), тридцять сім факторів Просвітлення (byang chub kyi chos) та
і т.д. Коротше, вона охоплює всі неймовірні якості знання, любові та здібностей
(mkhyen brtse nus), саме тому використовується назва "абхісамбодхіка" (mngon par
rdzogs par bang chub pa'i sku). Це Тіло є основою для виникнення всіх
унікальних якостей Стану Будди.

\subsubsection{Дхармакая}

Третє з Тел – це безтурботна Дхармакая (zhi ba chos kyi sku). Дхармакая не
є лише порожнечею, оскільки це не просто мудрість усвідомлення (rig ра ye shes).
Коли її розглядають з погляду самої Реальності, то Дхармакая – це Ваджракая, як
було пояснено вище, тоді як з погляду різноманіття її проявів та окремої
по суті, це Абхісамбодхікая, яка також була описана. Бо ж Дхармакая
визначається у цьому контексті? Вона не стала, оскільки вона за межами огляду
та осмислення. Але вона також не є відсутністю, оскільки вона є мудрістю,
розрізняє самоусвідомлення (so so rang rig pa'i ye shes). Вона не є тим і іншим або
а ні тим, ні іншим, оскільки ні сталість, ні повна відсутність не встановлюються.
Отже, вона має природу простору без центру та країв, як небо. У цьому
Простір дуже тонкий Мудрість змішується в одному смаку (го gcig). І хоча це
схоже на молодик, оскільки неочевидно (mi mngon ра), її пізнає аспект, у сенсі
внутрішньої ясності Мудрості (ye shes kyi nang gsal) є безперервним.\\
\\
Функції Дхармакаї, як сутність Мудрості, що розгортається (ye shes mched ра),
це ясність, яка спрямована назовні від медитативного стану. Як така, вона
функціонує як причина для Рупакаї (gzugs sku), для форм, які втілюються
проявляються для синів переможного (rgyal sras), які перебувають на різних рівнях
бодхісаттви, а також для звичайних істот. Сюди входять форми, які проявляються для
їх очей, просвітлена мова, яку вони чують, запах дисципліни благородних, смак
Дхарми, блаженне почуття медитативного зосередження, а також знання, пов'язане з
концептуальним аналізом, що оцінює явища (chos la 'jal ba). З цієї причини вона
розглядається як "Мудрість, яка спрямована всередину і не є двоїстою".
У тексті "Гуру магічної мережі (sgyu 'phrul Ыа та)" сказано:\\

\begin{verse}
Дхармака - це безоб'єктна основа для виникнення;
Вона є внутрішньо ясною і дуже тонкою мудрістю.\\
Ці три Тіла, внутрішньо ясний простір, дуже важко знайти.
\end{verse}

\subsubsection{Самбогака}

Четвертим [Тілом] є Самбхогакая (longs spyod). В "Етапах шляху" сказано:

\begin{verse}
Самопрояв усвідомлення,\\
Спонтанно досконале, виникає як маса променів світла,\\
Світи, небесні палаци, трони та прикраси.
\end{verse}

Як тут зазначено, просвітлені форми, наділені п'ятьма визначеностями
(nges ра lnga), виявляються із внутрішньої ясності, із Простору Реальності. Подібно
ясним проявам, які з'являються, коли промені сонця проникають у кристал, вони
проявляються природним чином, як втілення (bdag nyid) океану знаків та прикмет.
Такими є прояви самого Стану Будди, регентів, які є вчителями п'яти.
сімей Будди (rigs Inga'i ston ра). Вони проявляються подібно до порожніх форм і радіють
колесу безперервності, що постійно обертається (rgyun gyi 'khor lo). І оскільки вони
особливі (thun mong та yin ра), навіть ті, хто досяг десятого рівня не бачать їх, оскільки
вони ще не відкинули завіси (sgrib ра) повністю і їм ще належить брести око розуму
(bio mig), яке бачить всю різноманітність якостей, пов'язаних з тим, якою все є на
насправді і як це проявляються. У "Вищому потоці (rgyud Ыа та)" сказано:

\begin{verse}
Не те, про що можна сказати, що стосується абсолютного.\\
Не об'єкт концептуалізації, поза будь-якими прикладами.\\
Оскільки немає нічого вищого, не належить до існуючого чи спокою.\\
І навіть шляхетні не можуть охопити сферу переможних.
\end{verse}

Чудовий палац та інші елементи, пов'язані з досконалим місцем, виявляються
і виникають виходячи з цього базового ясного сяйва. Це можна порівняти з чистими
проявами, що виникають уві сні. Коли вони проявляються для тих, хто усунув усе
завіси, то вони не сприймаються як реальні та конкретні речі, що існують у якому-
то іншому місці. І це схоже на те, що різні порожні форми можуть виявлятися для
йогіна, чиї енергії увійшли до центрального каналу. Однак інші, що знаходяться в тому ж
місці все ще не можуть бачити ці форми.

Коли світло кристала втягнуте всередину, то воно залишається в основі для прояву цього
спектра. І так само три Тіла цієї внутрішньої ясності (nang gsal) окремо
присутні всередині базового простору тонкої мудрості. Наступний приклад
допомагає це зрозуміти. Коли сонце є як умова, промені світла проектуються з
кристала. І так само як елементи, що проявляються для нас, виникають об'єктивно,
зовнішня якість Мудрості сяє як форми, наділені всіма знаками та рисами
Стан Будди.

\subsubsection{Нірманака}

П'ятим [Тілом] є Нірманакая, яка може виявлятися як усе, що завгодно.
Тут ми маємо те, що називається "таємними вчителями, керманичами, які направляють
синів переможних, шляхетних та інших істот до острова безтурботності". Ці вчителі
є відображеннями Самбхогакаї, яка проявляється у сприйнятті вищих учнів.
І хоча вони виявляються як тотожні великі Самбхогакаї Простори,
що складається з елементів, що проявляються для себе, це не те, чим вони є на самому
справі. Якщо звернутися до прикладу, то різниця між цими двома подібна до різниці між
відображенням у дзеркалі та реальною річчю. Подібно до відображення, що проявляється нагадує
Самбхогакаю всіма своїми знаками та прикметами. Однак, ці чисті світи, почет та інші
подібні елементи проявляються для інших, а також знаходяться всередині проявів десятого
рівня. З цієї причини "Тантра сонця та місяця (nyi zla kha sbyor)" класифікує їх як
"наполовину Нірманакая та наполовину Самбхогакая".

Оскільки вони є природною еманацією того, що проявляється для самого
себе (rang snang gi rang bzhin 'phrul pa), вони відомі як "природна Нірманака чистих
сфер (zhing khams)". Ці сфери називаються "Неперевершеною", "Довершеною радістю",
"Славний", "Блаженний" (ще відомий як "Лотосовий пагорб"), а також "Здійснення
вищої активності". У цих п'яти сферах Вайрочана та інші Будди п'яти сімей вчать своє
чисте оточення учнів. На десятому рівні вони вчать їх природі п'яти сімей Будд, п'яти
Дхармам і п'яти трансформаціям, присвячуючи їх великим світлом (odzer chenpos dbang bskur).
В "Ілюзії актуалізації Просвітлення (sgyu 'phrul mngon byang)" сказано:

\begin{verse}
Досягши чистих рівнів,\\
А також з повної досконалості п'яти вчителів,\\
П'яти вищих Учень та п'яти Мудростей,\\
Переходять до сутності досконалого просвітлення.
\end{verse}

З цього стану виникають як "Нірманакая, яка приборкує істот". Форма
Нірманакаї походить з благого кармічного насіння (sa bon las) істот, і вона
проявляється так само, як місяць відкидає своє відображення у воді. І так само як місяць
має силу відкидати відображення, Самбхогакая має силу проектувати еманації,
які проявляються у сприйнятті тих, хто потребує керівництва. І так само як вода
є причинною ланкою, що дозволяє відображенню місяця з'являтися, ті, хто потребує
у посібнику, мають заслугу (bsod nams), що дозволяє виявлятися цим еманаціям. І
коли ці дві речі сходяться разом, відображення, що виявляються, виникають, щоб приборкати
істот необхідним способом, як і відображення місяця у воді з'являється без зусиль.\\
\\
Відповідно до карми кожного індивідуума ці еманації можуть набувати форми
вищих сфер, що проявляються вгорі, тварин, що проявляються збоку, або
істот пекла та парфумів, які проявляються внизу. Вони працюють заради блага шести класів
істот, які відчувають усі види страждання, відповідно до сприйняття кожного
рівня існування. Через це вони дбають про благополуччя інших, втілюючись як
істоти усвідомлення чи шестеро мудреців (rig pa'i skyes bu thub pa drug). Деякі
істоти підкоряються просвітленим тілом - дванадцятьма діяннями тощо,
здійснюваними тілом великої заслуги. Інші підкоряються просвітленою мовою -
різними колісницями, які є позамежними, істинно існуючими
звуками та словами. Багато хто підкоряється просвітленим розумом - шістьма формами
ясновидіння (mngon shes), такими як досконалі активності Самантабхадри. І, крім
того, багато хто підкоряється невимовною просвітленою активністю, що здійснюється
різними способами, прямо чи поступово. Як сказано в "Аватамсака-сутрі (phalро che)":

\begin{verse}
О, благородне дитя, втілення татхагати нічим не обмежені.
Вони дбають про благо істот, використовуючи ті форми,\\
Кольори та назви, які найкраще підходять для їх упокорення.
\end{verse}

З них, ті, що виявляють дванадцять просвітлених діянь, відомі як "вища
Нірманакая". Інші форми, які підкоряють інших своїм співчуттям, відомі як
"різноманітні нірманакаї". Ті, що народжуються таким чином, є еманаціями в
фізичних тілах, які безпосередньо допомагають істотам. Таким еманації діють
на благо інших тим, що втілюються, наприклад, як гігантська риба під час голоду, як
[чудове?] істота під час епідемії, або як кінь Аджанеябалаха, що поїхав у країну
демонів.\\
\\
Створена Нірманакая проявляється як фізичні об'єкти, зображення,
інкрустації, лотоси, що виконують бажання дерева, парки, сади, чудові палаци,
коштовності, кораблі, мости та світильники. Коротко, всі речі, які даруються на
благо живих істот є благословенними еманаціями. Все це виникає з
Простір і знову розчиняється в ньому. Спосіб, яким це відбувається, може
витримати найдокладніший аналіз. Всеведучий далі пояснює:

\begin{verse}
Коли немає утихомирюваного, утихомирювальне втягується в Простір;\\
Власний прояв Самбхагакаї втягується назад у Дхармакаю.
Так само як відображення місяця йде в небо, коли немає води, що відбиває,\\
Так як місяць знову зникає у просторі, коли настає час,\\
І так само як молодий місяць не росте і не зменшується,\\
Коли є утихомирювані істоти, вони поступово проявляються, як раніше.
Така спонтана присутність Плоду.
\end{verse}

Як тут зазначається, коли немає води для відображення місяця, воно природним чином
втягується назад у простір. І так само, коли немає істот, що підкоряються,
подібне Місяцю відображення Нірманакаї, Будда, який проявляється в їх сприйнятті, просто
розчиняється у стані безтурботності, всередині свого прояву Самбхогакаї.
Однак, незважаючи на використання цього прикладу, це не означає, що одна річ виникає
з чогось іншого, а потім розчиняється в ній.
\\
І так само Самбхогакая знову розчиняється у внутрішньому світлі Простору
Дхармакаї. Цей процес розглядається як "розчинення (thim ра) мудрості назад в
Простір", і це схоже на те, як молодий місяць залишається в стані внутрішнього
світла. Послідовники мадхьямаки пояснюють це як медитативну рівновагу (тпуат par
gzhag ра) всередині стану вищого припинення ('gog ра dam ра), яке виникає в
результаті безтурботного перебування у просторі явищ. Вони також вважають, що
корисна робота здійснюється у сприйнятті інших на підставі попередніх
побажань (smon lam). У "Вищої безперервності", з іншого боку, стверджується,
що хоча мудрість медитативної рівноваги не вагається, неймовірна кількість
корисного приноситься живим істотам завдяки постмедитативному стану. І тут
говориться:

\begin{verse}
Мудрістю вважається неконцептуальність,\\
А також наступне досягнення.\\
\end{verse}

\begin{siderules}
Оскільки зміст першого розділу докладно пояснюється в основному трактаті, його
легко зрозуміти. Далі, оскільки це детальне пояснення було б занадто
розлогим, я його не представив. І я також бачу, що тут немає жодних важких моментів,
які б потребували роз'яснення.
\end{siderules}

\subsection{Унікальна вистава мантри}

Що стосується чудового стану подвійної чистоти, в якому відкидання та
розуміння досягають своєї межі, говориться, що в Сутрі та Мантрі немає різниці між
Станом Будди. Така позиція мого вчителя, всезнаючого владики мови, Лонгченпи.
Однак, коли оцінюються якості простору щодо глибокого розуміння і
активностей ілюзорної Мудрості реалізованих істот, невимовні моменти можуть
бути зрозумілі точно, оскільки така сама природа їх знання, робити це з усім, з чим
стикаються.\\
\\
Однією такою людиною був безприкладний у поясненні таких тем Юнгтон Дордже
Пал. У традиції майстрів Зур, серед тих, хто практикував Великого Славного, він склав
трактат, що відокремлює ідеї Сутри про Стан Будди від ідей Мантри. Добре
відомо, що коли доходили до цього тексту, то навіть незрівнянні майстри пояснення
воліли виявляти скромність. Тим не менш, після ретельного розгляду,
можливим запропонувати аналіз цієї теми, на основі трактату цього майстра.\\
\\
Що стосується Дхармакаї, то він вчив, що можна зробити три поділи: поділ
щодо сутності (природи?) (ngo bo), поділ щодо характеристик
(mtshan nyid) та поділ щодо благословень (byin rlabs). Спочатку ми маємо
поділ щодо сутності. Дхармакая Причинної колісниці характеристик (rgyu
mtshan nyid kyi chos sku) є порожнеча, відсутність концептуальних проекцій (spros ра).
Дхармакая Колісниці мантр, з іншого боку, передбачає союз прояву та
порожнечі (snang stong zung 'jug), як стверджується в "Обширній ілюзії (rgyas ра)":

\begin{verse}
Одухотворений і неживий світ\\
Виявляються, хоч і не мають сутності.
\end{verse}

Що стосується другого поділу, то пояснюється, що Дхармакая Причинна
колісниці впадає в крайність порожнечі, тоді як Дхармакая Мантри є
нероздільністю прояви та порожнечі і, як така, вона не впадає в крайність. У
зв'язку з третім поділом він пояснює, що в результаті благословення Дхармакаї в
Причинної колісниці з'являються лише два Тіла-форми (gzugs sku), тоді як у традиції
Мантри благословення нероздільності прояву і порожнечі такі, що можуть
виникати п'ять Тіл та будь-які інші форми.\\
\\
Однак при детальному розгляді я не бачу жодної розумної причини проводити
відмінність щодо сутності Дхармакаї. Причина цього полягає в тому, що коли мова
йдеться про дуже тонку мудрість, яка спрямована всередину, але не притуплена, прояв
і порожнеча не ідентифікуються (ngos bzung med ра) як такі. Далі, цитата з
"Обширної ілюзії", яка цитується в даному контексті, насправді зовсім
відповідає розподілу, зробленому щодо характеристик Дхармакаї.

\\
\\
У другому розділі розбирається різниця між двома Тілами форм у цих традиціях,
і лише їх три. Що стосується першої, то в Причинній колісниці характеристик дві тіло-
форми виникають внаслідок причин та умов, тоді як у Колесниці мантр це не так. І
знову в "Обширній ілюзії" сказано:

\begin{verse}
Оскільки вони не залежать від причин та умов...
\end{verse}

Наступний розділ має два підрозділи, пов'язані з різницею щодо
Самбхогакаї та Нірманакаї. Що стосується Самбхогакаї, тобто дві різниці, і перша
пов'язана з різницею в тому, чим насолоджуються. Колісниця характеристик стверджує, що
насолоджуються позитивними факторами, тоді як негативними – ні. А також є відмінність,
що робиться на основі методів, завдяки яким насолоджуються цими факторами. Колісниця
характеристик не має методів використання негативних факторів, тоді як
Колісниця мантри має методи використання як позитивних, так і негативних.
факторів.
\\
Крім того, є дві різниці щодо Нірманакаї. Перша пов'язана з об'єктом -
істотами, які потребують керівництва. Нірмана Колісниці характеристик
може лише утихомирювати учнів з доброчесним характером, але з недобродійним.
Нірманак Колісниці мантри, з іншого боку, не робить таких поділів. Друга
різниця має відношення до методів, які використовуються для упокорення учнів. І
знову, Нірманакая Колісниці характеристик не має методів для утихомирення учнів з
негативними схильностями, тоді як Нірманак Колісниці мантри володіє методами
упокорення як добрих, так і злих учнів. Тому, підсумовується в тексті, Колесниця
мантри також може бути зрозуміла як найвища щодо нероздільності Ваджракаї
та Абхісамбодхікаї.\\

\begin{siderules}
У цьому розділі описуються особливі моменти методу Мантри щодо теми Стану
Будди. Є три традиції поділу Стану Будди в Колесницях Сутри та Мантри. У
першою стверджується, що крім різної довжини, два шляхи Сутри та Мантри по суті
однакові, оскільки обидва ведуть до Стану Будди, як до свого результату. У другій
стверджується, що без практики шляху Таємної мантри Ваджраяни неможливо
актуалізувати стан досконалого та повного Просвітлення. У третій стверджується,
що хоча на шляху Сутри можна розвиватися аж до одинадцятого рівня цієї традиції,
тобто до рівня Всеосяючого світла (kun tu 'od kyi sa), який тут розглядається як
Стан Будди, це не може відповідати справжньому стану Єдиного
Ваджрадхари (zung jug rdo rje 'chang), втілення океану Тіл і Мудростей, про які
йдеться у традиції Мантри. Іншими словами, ця традиція поділяє стан Будди в
Сутрі та Мантрі.\\
\\
Якщо ретельно дослідити, то останні дві позиції зводяться до одного й того самого. Що
стосується першої позиції, то необхідно прийняти те, що Стан Будди - одинадцятий
рівень Всесвітнього світла, шлях за межами медитації – є абсолютним
станом, що досягається на шляху Сутр. І навпаки, якщо шлях не веде до описуваного
результату, то цей шлях і його результат йдуть урозріз, подібно до шляхів до звільнення в
небуддійських традиціях. І це стає неприйнятним спотворенням Колісниці.
характеристик.
\\
Що ж до другої позиції, слід прийняти, що коли досягнуто Стан Будди,
пов'язане з шляхом Сутри, тим не менш, необхідно вступити на шлях Мантри, щоб
досягти стану Єдиного Ваджрадхари. З іншого боку, можна вважати, що
Стан Будди на шляху сутр є вершиною шляху та не використовувати шлях Мантри.
Проблема, однак, полягає в тому, що обмежуються двома абсолютними колісницями та
шлях Мантри не може бути навіть включений у три колісниці.\\
\\
Обидві позиції підтримуються багатьма мудрими та реалізованими індивідуумами. У
зокрема, текстову опору для поділу Стану Будди на Сутру та Мантру можна
знайти в "Магічній мережі (sgyu 'phrul)" і в "Вимовлення імен Манджушрі (mtshan brjod)".
А також пояснення цього приділяє велику увагу індійський вчитель
Буддхаджня нападу. У світлі того, що також багато інших текстових опор, це
положення не слід розглядати як безпідставну і лише умовну позицію.
\end{siderules}

\subsection*{Завершальні строфи та колофон}
\begin{verse}
Подібно до міцного ступеня до Простору \\
вищого та незмінного великого блаженства,\\
Прикрашений золотими коштовностями \\
детальних настанов про сенс тантр,\\
Це - збори добрих порад, \\
що прискорюють шлях до Аканіштхи,\\
Несе відповідальність за слова та зміст найвищої з Колесниць.
Втілення всіх Будд, Владика Уддіани, Махапандита Вімаламітра,
Всеведучий Владика Дхарми та сама життєва сила \\
Школи старих перекладів, бодхісаттви Зурчен та Зурчунг;\\
Коли більшість традицій навчання цих майстрів \\
були ледь живі у ці темні часи,\\
Вони були освітлені світлом, що виходить.
із сонця Ранджунга Дордже Кхенце.\\
Ця скарбниця виконує бажання коштовностей \\
очищає сутнісні та ключові настанови,\\
Збори тантр та садхани, повністю та безпомилково.\\
Ті, хто прямують цими стопами і прямують до п'яти Тіл,\\
Безперечно, є незаперечно щасливими гостями.\\
Безтурботна річка цих початкових зборів заслуги,\\
Супроводжувана послідовністю хвиль трьох переваг,\\
Здійснює передачу знання Манджукумари.
Завдяки цьому, нехай всі досягнуть рівня Самантабхадри!
\end{verse}

\small
Чагсам Рігдзін – найвища реінкарнація Еше Лхундруба Палзангпо і той, хто
досяг благого ока розуму, що широко бачить те, чого татхагати вчать так добре. І все ж
він перебуває без марнославства щодо навчань і тих, хто підтримує їх з великою
турботою та повагою. З великою завзятістю він просив, щоб я склав цей текст
"Сходи в Аканіштху: Настанова за Стадією Зародження та Йогою божества", супроводжуючи
своє прохання підношенням одягу вишитим золотом і пари різнокольорових шарфів.
У відповідь на це прохання бутон лотоса мого розуму миттєво розкрився у світлі чудових
променів любові та мудрості, які були випущені славетним Падмасамбхавою та його дружиною.
В результаті розуміння всіх явищ виникло природно, і я зміг безстрашно залишатися
перед лицем справжньої істини. Я, Рангджунг Дордже Джігме Лінгпа, практикуючий
Великої досконалості, також відомий як Лонгчен Намке Налджор, потім склав цей
текст у Церінг Джонг. Він був записаний посеред лісу просвітлення в ретритному центрі,
відомому як Падма Осел Тегчог Лінг, у чудовому місці для духовної практики, у храмі,
п'яти сімей, що охороняється татхагатами. Під час написання цього пояснення мені було
тридцять дев'ять років. Роботу було завершено у четвер, у місячний місяць танцюриста (gar mkhan
gyi khyim zla), у рік свині (1767), під час сходження будинку лева (seng ge'i ta tal la 'char
ba).

\begin{center}
БЛАГО БЛАГО БЛАГО
\end{center}

\begin{siderules}
Якщо добра удача слухання лише єдиного слова вищих навчань,\\
Ті, хто міститься в лінії Всеведучих, виходить за межі мислимого,\\
Тоді ці певні ідеї, що ґрунтуються на поясненні, полеміці та складанні,\\
З'являються внаслідок сили накопичення досягнення багато кальп.\\
\\
І хоча сам я не знайшов не найменшої впевненості.
За допомогою вивчення, споглядання та медитації,\\
Я написав це, щоб ознайомитися з цими вченнями,\\
І щоб допомогти небагатьом удачливим та доброзичливим.\\
\\
Знаючі вчені, які багато чого вивчили,\\
Йоги, досвідчені у медитації,\\
І ті щасливці, які мають сутнісні настанови,\\
Будь ласка, дослідіть цей підхід і усуньте нечисте своєю любов'ю.\\
\\
Так, я закликаю всіх науковців, чиї описи сотень текстів\\
Може прояснити пояснення Всезнаючого нашої традиції.
Відсікти помилки, які затемнюють завісами\\
Незнання, нерозуміння та часткового розуміння.\\
\\
Якщо тут є ясне пояснення, то я присвячую.
Нагромадження цієї заслуги процвітання Трьох Коштовностей,\\
А також тим, хто прояснює сутнісні вчення Могутого,\\
А також постійно підтримує, пояснює та практикує їх.\\
\\
У відповідь на прохання Лхундруба Дордже, що володіє ясним розумінням, уродженим і\\
придбаним в результаті навчання, я, безтурботний бродяга Any, записав те, що прийшло на \\
розум. Нехай це принесе велике благо!
\end{siderules}

\begin{center}
САРВА МАНГАЛАМ
\end{center}