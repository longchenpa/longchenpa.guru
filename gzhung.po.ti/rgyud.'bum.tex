\documentclass{article}
\usepackage{hyperref}
\usepackage[english,ukrainian]{babel}
\usepackage[utf8]{inputenc}

\begin{document}

\title{Ньїнгма Г'юбум}
\author{Максим Сохацький $^1$}
\date{ $^1$ Майстер лінії Лонгчен Нінгтік. }

\maketitle

\begin{abstract}Дослідження тантр древніх. \\
\textbf{Ключові слова}: Ваджраяна.
\end{abstract}

\tableofcontents

\vspace{1cm}

Зібрання Тантр Древніх було перекладено в ранній період розповсюдження буддизму в Тибеті.
Обидва періоди (snga dar і phyi dar) відрізняються своїми перекладами тантр: ранні
переклади (snga 'gyur), виконані під керівництвом Вайрочани та його сучасників (8–9 століття),
і пізні переклади (phyi 'gyur), розпочаті під керівництвом Рінчена Зангпо (rin chen bzang po).
Школа Ньїнгма визнає ранні переклади (snga 'gyur) як найцінніші тексти нетибетських рукописів,
що датуються першим періодом становлення буддизму та його розквітом. Вони дали змогу подивитися
на Дзогчен, не просто як на особливу стадію Дзогрім, а як на самостійну школу мислення в
будизмі (істинну практику Йогачари).

\section{Унікальна лінія}

\subsection{Колекція Вайрочани}

Вайро Г’юбум — це найдавніше Зібрання Тантр Древніх (Ньїнгма Г’юбум).
Вона була складена і перекладена тибетським майстром восьмого століття Вайрочаною.
Відтворено з рідкісного рукопису, що належав Токдену Рінпоче з Гандонга. Байро Г’юбум,
що складається з 8 томів — колекція ньїнгмапінської тантри, виявлена у кочівників північного
Ладакха. Колекція складається з 8 томів і 96 текстів \footnote{\url{https://longchenpa.guru/gzhung.po.ti/rgyud.'bum/bai.ro.txt}}.
Академічне дослідження виконане Метью Капштейном \footnote{\url{https://longchenpa.guru/gzhung.po.ti/rgyud.'bum/bai.ro.pdf}}.
Колекція Вайрочани є тренувальною колекцією для аналізу дослідників, що йдуть стопами Вайрочани.

\subsection{Колекція Джикме Лінгпи}

Колекція Джігме Лінгпи (1729–1798) — це попередня версія до Деге, підготовлена до того,
як Геце Махапандіта видав кселофонічну версію. Складає 26 томів, 388 текстів. Академічне дослідження.

\subsection{Колекція Деге}

Створена в 1794–1798 роках у районі Деге, Східний Тибет, є першим Зібранням Тантр Древніх, виконаним у вигляді дерев’яної гравюри. Її складання доручила цариця Деге Цеванг Лхамо (tshe dbang lha mo). Тертон Ратна Лінгпа (1403–1471) відіграв важливу роль у складанні першого звернення Нінгма Г'юбум. Джігме Лінгпа зібрав тексти Ньїнгма, які стали рідкісними, починаючи з тантр Нінгма, що зберігалися в рукописній колекції монастиря Міндролінг. Ця колекція тантр Нінгма призвела до накопичення «Колекції тантр Нінгма» – Нінгма Г'юбум (rnying ma rgyud 'bum), для якої Гетсе Махапандіта написав каталог, вичитав та організував його друк, замовивши дорогий та трудомісткий проект «друку» на дерев'яних дощечках. Різблення було здійснено завдяки патронату королівської родини Кхама «Деге», яка прихильно ставилася до Джігме Лінгпи та шанувала його. Кількість текстів у колекції Деге — 448. Академічне дослідження.

Існує чотири копії цих рукописів:
1) Каталогізована Орофіно в 1998 році, зберігається в Tucci Fund Collection в IsIAO Бібліотеці Сходу в Римі,
    подарована Далай-ламою XIV Тензіном Г’яцо (bstan 'dzin rgya mtsho) у 1949 році в Лхасі.
2) Переведена до Катманду з Національної бібліотеки в 1992 році.
3) Знаходиться в приватній бібліотеці Гурме Дордже, куплена в Деге в 1989 році.
4) Відскановане видання (1794 і 1798 рр.) каталогізоване Гарсоном, Хіллом, Возом і
   Вайнбергом під керівництвом Джермано. Перевидання Чакунг Джігме Вангдрак Рінпоче
   в 2025 році (робота тривала 13 років).

\section{Лінія південно-центрального Тибету}

\subsection{Колекція Рігдзіна Цеванга Норбу}
Рігзін Цеванг Норбу Г’юбум (rig 'dzin tshe dbang nor bu rgyud 'bum) ---
це одна з редакцій Зібрання Тантр Древніх (rnying ma rgyud 'bum). Складена
Рігдзіном Цеванг Норбу наприкінці 18-го століття. Рукопис із ілюмінаціями,
що складається з 33 томів, з яких 2 не збереглися, походить із Південно-Центрального
Тибету (Катхок) і має спільного предка з Тінгк’є та Катманду.

\subsection{Колекція Тіньк'є}
Тінгк’є Г’юбум (gting skyes rgyud 'bum) — це одна з редакцій Зібрання Тантр Древніх.
Відтворена Ділго Кх’єнце Рінпоче у 1974–1975 роках із рукописів монастиря Тінгк’є.
Складається з 36 томів і вважається більш повною версією,
опублікованою \footnote{\url{https://longchenpa.guru/gzhung.po.ti/rgyud.'bum/gting.skyes.txt}}
в Нью-Делі та Тхімпху.

Редакція Південно-Центрального Тибету представлена рукописами Тінгк’є (gting skyes),
Рігзін Дже (rig 'dzin rje, Waddell), Нубрі (skyid grong) та рукописом із Катманду.
Про їхнє походження в Тибеті відомо мало. Вони є або копіями редакції Деге,
або редакціями, що передували їй. Деякі тексти Тінгк’є унікальні або змінені
порівняно з Деге. Кількість текстів — 447. Академічне
дослідження\footnote{\url{https://longchenpa.guru/gzhung.po.ti/rgyud.'bum/sde.dge.pdf}}.

\subsection{Колекції Нгаванга Лхундруба Дракпи}
Обидві спочатку складалися з 37 томів: 1) комплект Нубрі (Nu),
 ілюмінованого рукописного видання, яке було виготовлено в Дракар
Тасо і наразі зберігається в Нубрі, та 2) комплект НАК (Na), ілюмінованого
рукописного видання, ймовірно, виготовленого в районі Солу Кхумбу і наразі
зберігається в Національному архіві в Катманду (неповне). Як зазначалося
раніше, обидва були виготовлені на початку 19 століття: набір Нубрі був
виготовлений у Дракар Тасо за наказом Дракар Тасо Тулку Чокі Ванпуг (1775–1837)
протягом 1813–1814 років, а набір НАК, ймовірно, в районі Солу Кхумбу невдовзі
після набору Нубрі за наказом одного Орджена Трінлас Тензин, учня Дракар Тасо Тулку.
Академічне дослідження.

\subsection{Колекція Тубтена Ньїми}
Компіляція Г’юбум 2015 року (2015 rgyud 'bum) — це одна з редакцій Зібрання Тантр Древніх. Складена у 2015 році Тубтеном Ньїмою та Дродул Дорджем за підтримки bka' bstan dpe sdur khang. Складається з 49 томів (46 томів текстів + 3 томи каталогів). Цифрова компіляція на основі Деге, Цамдрак, Тінгк’є та Гангтенг, опублікована в Пекіні. Академічне дослідження

\subsection{Колекція Дудула Дордже}
Палцек (dpal brtsegs rgyud 'bum) — це колекція Дудула Дордже яка містить 58 томів і 975 текстів. Перевидано Алаком Зенкаром Рінпоче. MW1KG14783 (rnying rgyud/ dpal brtsegs bod yig dpe rnying zhib 'jug khang/). Це основна робоча колекція сучасних дослідників.

\section{Бутанська лінія}

\subsection{Колекція Цамдрак}
Цамдрак Г’юбум (mtshams brag rgyud 'bum) — це одна з редакцій Зібрання Тантр Древніх. Рукопис 18-го століття, опублікований у 1982 році Національною бібліотекою Бутану. Складається з 46 томів, є найоб’ємнішим виданням, знайденим у монастирі Цамдрак у Західному Бутані. Включає Kunjed Gyalpo як перший текст. Є фотокопією, зробленою в 1982 році в монастирі Цамдрак у Бутані. Оригінальний автор невідомий, але відомо, що видання створено в 1728–1748 роках із оригіналу Пунакха за наказом Спрулску Нгагванг Друкпа (sprul sku ngag dbang 'brug pa, 1682–1748). Ця редакція дуже схожа на Гангтенг В (sGang steng B), також 46 томів, нещодавно каталогізований Кентвелл, Маєром, Ковалевським та Ахардом. Цамдрак є унікальним і містить найбільшу кількість текстів — 939. Академічне дослідження.

\subsection{Колекція Драмеце}
Гангтенг А (sgang steng A) та Драмеце (sbra me'i rtse) ще не каталогізовані, їх походження пов’язане з Цултімом Дордже (tshul khrims rdo rje, 1598–1669) та Нгагванг Кунзанг Дордже (ngag dbang kun bzang rdo rje, 1680–1723). Обидві копії, ймовірно, створені приблизно в 1640–1650 роках (Кентвелл, Маєр). Малоймовірно, що Цамдрак копіювався з Гангтенг В чи навпаки; швидше, вони походять від іншого джерела, можливо, Гангтенг А або Драмеце. Академічне дослідження.

\subsection{Колекція Гангтенга Тулку}
Гангтенг Г’юбум (sgang steng rgyud 'bum) — це одна з редакцій Зібрання Тантр Древніх. Рукопис 18-го століття, походить від Лхалунг. Бутанські версії, досліджені Кентвелл та Маєром. Детальне дослідження структури цього видання виконали Кеті Кентвелл, Роб Мейер, Майкл Ковалевський, Жан-люк Ахард. Забрання містить 46 томів і 924 тексти. Академічне досдідження.

\end{document}
